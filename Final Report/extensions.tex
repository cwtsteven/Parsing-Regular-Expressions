\chapter{Further Extensions}
\par In this section, we will discuss two possible extensions to
our project: 1) Myhill-Nerode Theorem and 2) Pumping Lemma. In these two
theorems, they both use the translation of regular expressions to DFA
in their arguments. Therefore, they can be built on top of our
formalisation. 


\section{Myhill-Nerode Theorem}
\par Given a language \(L\), and a pair of
strings \(u\) and \(v\), we define a relation \(R_L\) on strings such
that \(u R_L v\) if and only if for any string \(z\), \(uz \in L
\Leftrightarrow vz \in L\). The relation \(R_L\) is an equivalence
relation and thus it divides the set of strings into some equivalence
classes.
\par In general, the Myhill-Nerode Theorem states that a language
\(L\) is regular if and only if \(R_L\) has a finite number of
equivalence classes, and furthermore the number of states in the
minimal DFA translated from \(L\) is equal to the number of
equivalence classes in \(R_L\). The theorem is often used to prove a
language to be regular. 

\par In the proof, if \(L\) is a regular language, then a DFA is
translated from \(L\) to prove that \(R_L\) has finite equivalence classes. For the opposite direction, suppose \(R_L\)
has finite equivalence classes, then a DFA is also constructed from
the relation. By proving the DFA recognises \(L\), \(L\) is proved to
be regular. Therefore, the theorem can
be proved under the environment of our formalisation as we have
already formalised the translation from regular languages to DFA and its correctness proof. 


\section{Pumping Lemma (for regular languages)}
\par In general, the
Pumping Lemma states that if \(L\) is a regular language, then there
must exist an integer \(n \geq 1\) depending on \(L\) such that every
string \(w \in L\) of length at least \(n\) can be divided into three
sub-strings, i.e. \(w = xyz\) where
\begin{enumerate}[nolistsep]
  \item \(|y| \geq 1\);
  \item \(|xy| \leq n\);
  \item for all \(i \geq 0\), \(xy^iz \in L\).
\end{enumerate}
\par \(y\) is the sub-string that can be pumped (removed or repeated
any number of times) and the result string is always in \(L\). The
lemma is often used to prove a language to be non-regular. In the proof, a DFA is constructed from \(L\) to prove that there
must exist some loops in the path of \(w\). Therefore, it can be
proved by using our formalisation. 