\section{Minimising DFA}
\par There are two procedures in minimising a DFA: 1) removing all the
unreachable states to construct a RDFA and 2) perform quotient
construction on the RDFA to build a MDFA. 

\subsection{Removing unreachable states}
\par Let us begin by defining the
reachability of a state. The following definitions and proofs are
defined under the module \textbf{Remove-Unreachable-States} in
\textbf{Translation/DFA-MDFA.agda}. 

\begin{defn}
\noindent For a given DFA, \((Q,\ \delta,\ q_0,\ F)\), we say that a state \(q\) is reachable if and
only if there exists a string \(w\) that can take \(q_0\) to \(q\), 
\end{defn} 

\par It is defined in Agda as follow:
\begin{lstlisting}[mathescape=true,xleftmargin=.25\textwidth]
Reachable : Q $\to$ Set
Reachable q = $\exists$[ w $\in$ $\Sigma^*$ ] (q$_0$ , w) $\vdash^*$ (q , [])
\end{lstlisting}

\begin{defn}
\noindent For a given DFA, \((Q,\ \delta,\ q_0,\ F)\), we define \(Q^R\) as a
subset of \(Q\) such that \(Q^R\) contains all and only the reachable states in \(Q\). 
\end{defn}

\par The set \(Q^R\) is defined in Agda as follow:
\begin{lstlisting}[mathescape=true,xleftmargin=.25\textwidth]
data Q$^R$ : Set where
  reach : $\forall$ q $\to$ Reachable q $\to$ Q$^R$
\end{lstlisting}

\par There are some problems regarding this formalisation of \(Q^R\),
they will be discussed in Chapter 6 in details. Now, we can construc the
translation of DFA to RDFA which is defined 
as the function \textbf{remove-unreachable-states} in the same file. 

\begin{defn}
\label{defn:unreachable}
\noindent For any given DFA, \((Q,\ \delta,\ q_0,\ F)\), its
translated RDFA will be \((Q^R,\ \delta,\ q_0,\ Q^R \cap F)\). 
\end{defn}

\par Now, let us prove the correctness of the translation by proving
that their accepting languages are equal. The correctness proof is
defined as the function \mmb{L^D\!\approx\! L^R} under the module
\textbf{Remove-Unreachable-States-Proof} in \textbf{Correctness/DFA-MDFA.agda}. 

\begin{thm}
\noindent For any DFA, its accepting language is equal to
the language accepted by its translated RDFA, i.e. \(L(\)DFA\()
= L(\)translated RDFA\()\). 
\end{thm}

\begin{proof}
\noindent Let the DFA be \(D = (Q,\ \delta,\ q_0,\ F)\) and is
translated RDFA be \(R = (Q^R,\ \delta,\ q_0,\ Q^R \cap F)\) according
to \autoref{defn:unreachable}. To prove the theorem, we have to prove
that \(L(D) \subseteq L(R)\) and \(L(D) \supseteq L(R)\). For any
string \(w\), 

\par \noindent 1) if \(D\) accepts \(w\), then \(w\) must be able to
take \(q_0\) to an accepting state \(q\). This implies that all the states in
the path must be reachable; therefore, this path is also valid in
\(R\) and thus \(R\) accepts \(w\); 

\par \noindent 2) if \(R\) accepts \(w\), then the path from \(q_0\)
to the accepting state \(q\) must be also valid in \(D\); therefore,
\(D\) also accepts \(w\). 
\end{proof}


\subsection{Quotient construction}
\par Now, we need to perform quotient construction on the newly
constructed RDFA. The following definitions and proofs are defined
under the module \textbf{Quotient-Construct} in
\textbf{Translation/DFA-MDFA.agda}. Now, let us begin by defining a
binary relation on states that will be used to construct the quotient set. 

\begin{defn}
\noindent Suppose we have a DFA \((Q,\ \delta,\ q_0,\ F)\), then for any two states \(p\) and \(q\), we say
the \(p \sim q\) if and only if for any string \(w\),
\(w\) cannot distinguish \(p\) and \(q\), i.e. \(p \sim q = \forall w.\
\delta^*(p,w) \in F \Leftrightarrow \delta^*(q,w) \in F\)
\end{defn}

\par It is easy to show that the relation is an equivalence
relation. Now, we have to prove that \(p \sim q\) is decidable. One
possible method is to use the Table-filling algorithm. However, the
formalisation of this algorithm and its correctness proofs has not
been completed. Therefore, in the following parts, we will assume that
\(p \sim q\) is decidable. Now, let us construct the quotient set by using the above
relation of states. The quotient set is defined in the
\textbf{Quotient.agda}. 

\begin{defn}
\noindent For a state \(p\) of a given DFA, its equivalence
class is a subset of all indistinguishable states of \(p\), i.e. \(\llangle p
\rrangle = \{q\ |\ p \sim q\}\). 
\end{defn}

\par From the above definition, we observe that a equivalence class
is a subset of the set of states of a given DFA. The corresponding formalisation in
Agda is as follow, note that \(Dec\hyphen\!\sim\) is the decidability of
\(\sim\):
\begin{lstlisting}[mathescape=true,xleftmargin=.2\textwidth]
$\llangle$_$\rrangle$ : Q $\to$ DecSubset Q
$\llangle$ p $\rrangle$ q with Dec$\hyphen\!\sim$
... | yes _ = true
... | no  _ = false
\end{lstlisting}

\begin{defn}
\noindent For a given DFA, \((Q,\ \delta,\ q_0,\ F)\), we define
\(Q/\!\sim\) as the set of all equivalence classes of
\(\sim\) on \(Q\). 
\end{defn}

\par It is defined in Agda as follow: 
\begin{lstlisting}[mathescape=true,xleftmargin=.2\textwidth]
data Q/$\!\sim\!$ : Set where
  class : $\forall$ qs $\to$ $\exists$[ q $\in$ Q ] (qs $=^d$ $\llangle q \rrangle$) $\to$ Quot-Set
\end{lstlisting}

\par Now, we can define the translation of RDFA to MDFA which is
defined as the function \textbf{quotient-construction} at the bottom of \textbf{Translation/DFA-MDFA.agda}. 
\begin{defn}
\label{defn:quotient}
\noindent For any given RDFA, \((Q,\ \delta,\ q_0,\ F)\), its
translated MDFA will be \((Q/\!\sim,\ \delta',\ \llangle q_0 \rrangle,\ F')\) where
\begin{itemize}[nolistsep]
\item \(\delta'(q,a) = \llangle \delta(q,a) \rrangle\); and
\item \(F' = \{\llangle q \rrangle\ |\ q \in F\}\).
\end{itemize}
\end{defn}

\par Now, before proving the translation is correct, we have to first prove
the following lemmas. The theorems and proofs below can be found in
the module \textbf{Quotient-Construction-Proof} in \textbf{Correctness/DFA-MDFA.agda}. 

\begin{lem}
\label{lem:rdfa<mdfa}
\noindent Let a RDFA be \(R = (Q,\ \delta,\ q_0,\ F)\) and its
translated MDFA be \(M = (Q/\!\sim,\ \delta',\ \llangle q_0 \rrangle,\
F')\) according to \autoref{defn:quotient}. For any state \(q\) in \(Q\), if a string \(w\) can take \(q\) to another state \(q'\) with \(n\) transitions
in \(R\), then \(w\) can take \(\llangle q \rrangle\) to \(\llangle q'
\rrangle\) with \(n\) transitions in \(M\). 
\end{lem}

\par The proof is defined as the function \textbf{lem\(_1\)}. 

\begin{proof}
\noindent We can prove the lemma by induction on \(n\).
\par \noindent \textbf{Base case.}\quad If \(n = 0\), then \(q = q'\)
and \(w = \epsilon\). It is obvious that the statement holds.

\par \noindent \textbf{Induction hypothesis.}\quad For any state \(q\)
in \(Q\), if a string \(w\) can take \(q\) to another state \(q'\)
with \(k\)  transitions
in \(R\), then \(w\) can take \(\llangle q \rrangle\) to \(\llangle q'
\rrangle\) with \(k\) transitions in \(M\). 

\par \noindent \textbf{Induction step.}\quad If \(n = k + 1\), then
there must exist a string in the form of \(aw\) that can take \(q\)
to \(q'\) with \(k + 1\) transitions. This implies that there exists a
state \(p\) such that \(p = \delta(q,a)\) and \(w\) can take \(p\) to \(q'\) with \(k\) transitions. By
induction hypothesis, \(w\) can also take \(\llangle p \rrangle\) to \(\llangle q'
\rrangle\) with \(k\) transitions in \(M\). Furthermore, \(p =
\delta(q,a)\) implies that \(\llangle p \rrangle = \llangle
\delta(q,a) \rrangle = \delta'(\llangle q \rrangle,a)\); Therefore, \(aw\)
can take \(\llangle q \rrangle\) to \(\llangle q' \rrangle\) with \(k
+ 1\) transitions in \(M\) and thus the statement is true. 
\end{proof}

\begin{lem}
\label{lem:rdfa>mdfa}
\noindent Let a RDFA be \(R = (Q,\ \delta,\ q_0,\ F)\) and its
translated MDFA be \(M = (Q/\!\sim,\ \delta',\ \llangle q_0 \rrangle,\
F')\) according to \autoref{defn:quotient}. For any string \(w\) and
any state \(q\) in \(Q\), \(\delta'^*(\llangle q \rrangle,w) = \llangle \delta^*(q,w) \rrangle\). 
\end{lem}

\par The proof is defined as the function \textbf{lem\(_2\)}. 

\begin{proof}
\noindent We can prove the lemma by induction on \(w\). 

\par \noindent \textbf{Base case.}\quad If \(w = \epsilon\),
\(\delta'^*(\llangle q \rrangle,\epsilon) = \llangle q \rrangle =
\llangle \delta^*(q,\epsilon) \rrangle\) and thus the statement is
true. 

\par \noindent \textbf{Induction hypothesis.}\quad For a string \(w'\) and
any state \(q\) in \(Q\), \(\delta'^*(\llangle q \rrangle,w') = \llangle \delta^*(q,w') \rrangle\). 

\par \noindent \textbf{Induction step.}\quad If \(w = aw'\), then
\(\delta'^*(\llangle q \rrangle,aw') = \delta'^*(\delta'(\llangle q
\rrangle,a),w')\). Since, \(\delta'(\llangle q \rrangle,a) = \llangle
\delta(q,a) \rrangle\); therefore, \(\delta'^*(\delta'(\llangle q
\rrangle,a),w') = \delta'^*(\llangle
\delta(q,a) \rrangle,w')\). By induction hypothesis,
\(\delta'^*(\llangle \delta(q,a) \rrangle\,w') = \llangle
\delta^*(\delta(q,a),w) \rrangle = \llangle
\delta^*(q,aw) \rrangle\) and thus the statement is true. 
\end{proof}

\par Now, we can prove the correctness of the translation by proving
that their accepting languages are equal. The correctness proof is
defined as the function \mmb{L^R\!\approx\! L^M} in
\textbf{Correctness.agda} while the detail proofs are defined in \textbf{Correctness/DFA-MDFA.agda}

\begin{thm}
\noindent For any RDFA, its accepting language is equal to
the language accepted by its translated MDFA, i.e. \(L(\)RDFA\()
= L(\)translated MDFA\()\). 
\end{thm}

\begin{proof}
\noindent Let the RDFA be \(R = (Q,\ \delta,\ q_0,\ F)\) and its
translated MDFA be \(M = (Q/\!\sim,\ \delta',\ \llangle q_0 \rrangle,\
F')\) according to \autoref{defn:quotient}. To prove the theorem, we
have to prove that \(L(R) \subseteq L(M)\) and \(L(R) \supseteq
L(M)\). For any string \(w\),

\par \noindent 1) if \(R\) accepts \(w\), then \(w\) must be able to
take \(q_0\) to an accepting state \(q\) in \(R\). By \autoref{lem:rdfa<mdfa},
\(w\) must be able to take \(\llangle q_0 \rrangle\) to \(\llangle q
\rrangle\) in \(M\). Since \(q \in F\); therefore, \(\llangle q
\rrangle \in F'\) and thus \(M\) accepts \(w\). 

\par \noindent 2) if \(M\) accepts \(w\), then \(w\) must be able to
take \(\llangle q_0 \rrangle\) to an accepting state \(\llangle q
\rrangle\). Then by \autoref{lem:dec_iff}, \(\delta'*(\llangle q_0
\rrangle,w) \in F'\). Furthermore, by \autoref{lem:rdfa>mdfa},
\(\delta'*(\llangle q_0 \rrangle,w) = \llangle \delta(q_0,w)
\rrangle)\); therefore, \(\llangle \delta(q_0,w) \rrangle \in F'\) and
thus \(delta(q_0,w) \in F\). By \autoref{lem:dec_iff2}, \(w\) can take
\(q_0\) to an accepting state \(q\) in \(R\); therefore, \(R\) accepts
\(w\). 
\end{proof}

\par We also need to prove that the translated MDFA is minimal, but
first, we have to define what is a minimal DFA. 


\section{Minimal DFA}
\par The definition of minimal DFA can be found in
\textbf{MinimalDFA.agda}. In order for a DFA to be minimal,
it must satisfy two criteria: 1) every state must be reachable and 2) the
states cannot be further reduced. 

\par Criteria (1) is defined as the function
\textbf{All-Reachable-States}. It is defined as follow: 
\begin{lstlisting}[mathescape=true,xleftmargin=.1\textwidth]
All-Reachable-States : DFA $\to$ Set
All-Reachable-States dfa = $\forall$ q $\to$ Reachable q
\end{lstlisting}

\par For criteria 2), we have to first introduce a binary relation of states. 
\begin{defn}
\noindent Suppose we have a DFA, \((Q,\ \delta,\ q_0,\ F)\), for any two states \(p\) and \(q\), we say that a string
\(w\) can distinguish \(p\) and \(q\) if and only if exactly one of
\(\delta^*(p,w)\) and \(\delta^*(q,w)\) is an accepting state,
i.e. \(\delta^*(p,w) \in F \wedge \delta^*(q,w) \notin F \vee
\delta^*(p,w) \notin F \wedge \delta^*(q,w) \in F\). 
\end{defn}

\begin{defn}
\noindent For any two states \(p\) and \(q\) in a given DFA, we say that \(p \nsim
q\) if and only if there exists a string \(w\) that can distinguish
\(p\) and \(q\). 
\end{defn}

\par It is defined in Agda as follow:
\begin{lstlisting}[mathescape=true,xleftmargin=.1\textwidth]
_$\nsim$_ : Q $\to$ Q $\to$ Set
p $\nsim$ q = $\exists$[ w $\in$ $\Sigma^*$ ] 
     ($\delta^*$ p w $\in^d$ F $\times$ $\delta^*$ q w $\notin^d$ F $\uplus$ $\delta^*$ p w $\notin^d$ F $\times$ $\delta^*$ q w $\in^d$ F)
\end{lstlisting}

\begin{defn}
\noindent For a given DFA, it is irreducible if and only if for any
two states \(p\) and \(q\), if \(p\) is not equal to \(q\), then \(p \nsim q\). 
\end{defn}

\par It is defined in Agda as follow:
\begin{lstlisting}[mathescape=true,xleftmargin=.1\textwidth]
Irreducible : DFA $\to$ Set
Irreducible dfa = $\forall$ p q $\to$ $\neg$ p $\approx$ q $\to$ p $\nsim$ q
\end{lstlisting}

\par Now, we can define what is a minimal DFA. 

\begin{defn}
\noindent For a given DFA, it is minimal if and only if all its states
are reachable and it is irreducible. 
\end{defn}

\par It is defined in Agda as follow:
\begin{lstlisting}[mathescape=true,xleftmargin=.1\textwidth]
Minimal : DFA $\to$ Set
Minimal dfa = All-Reachable-States dfa $\times$ Irreducible dfa
\end{lstlisting}

\par Now, let us prove that any MDFA is minimal by proving that all the states in MDFA are reachable and the MDFA
is irreducible. The proofs are defined under the module
\textbf{Reachable-Proof} and \textbf{Minimal-Proof} in
\textbf{Correctness/DFA-MDFA.agda}. 

\begin{thm}
\label{thm:all_reach}
\noindent For any MDFA, all its states are reachable. 
\end{thm}

\begin{proof}
\noindent Since for any state in a RDFA is reachable, then for any
state \(q\) in RDFA, there must
exist a string \(w\) that can take \(q_0\) to \(q\). By
\autoref{lem:rdfa<mdfa}, \(w\) can also take \(\llangle q_0
\rrangle\) to \(\llangle q \rrangle\) and thus the statement is true. 
\end{proof}

\par To prove that MDFA is irreducible, we have to first prove
that for any two states \(p\) and \(q\), \(\neg (p \sim q\)) if and only if
\(p \nsim q\). 

\begin{lem}
\label{lem:sim_nsim1}
\noindent For any two states \(p\) and \(q\) in a DFA and a string
\(w\), \(\delta^*(p,w) \in F \Leftrightarrow \delta^*(q,w)\) if and
only if \(not\ exactly\ one\ of\ \delta^*(p,w)\ and\ \delta^*(q,w)\
is\ an\ accepting\ state\), i.e. \((\delta^*(p,w) \in F \Leftrightarrow
\delta^*(q,w)) \Leftrightarrow \neg (\delta^*(p,w) \in F \wedge \delta^*(q,w) \notin F \vee
\delta^*(p,w) \notin F \wedge \delta^*(q,w) \in F)\). 
\end{lem}

\begin{proof}
\noindent We have to prove for both directions. 

\par \noindent 1) Case \((\delta^*(p,w) \in F \Leftrightarrow
\delta^*(q,w) \in F) \Rightarrow not
exactly one of \delta^*(p,w) and \delta^*(q,w) is
an accepting state\): suppose exactly one of \(\delta^*(p,w)\) and
\(\delta^*(q,w)\) is an accepting state, for example, \(\delta^*(p,w)
\in F \wedge \delta^*(q,w) \notin F\). This contradicts to the
assumption that \(\delta^*(p,w) \in F \Leftrightarrow \delta^*(q,w) \in F\). Therefore, not
exactly one of \(\delta^*(p,w)\) and \(\delta^*(q,w)\) is
an accepting state. The same argument also applies for the case when \(\delta^*(p,w)
\notin F \wedge \delta^*(q,w) \in F\).

\par \noindent 2) Case \(not\ 
exactly\ one\ of\ \delta^*(p,w)\ and\ \delta^*(q,w)\ is\ 
an\ accepting\ state\ \Rightarrow (\delta^*(p,w) \in F \Leftrightarrow
\delta^*(q,w) \in F)\): if \(\delta^*(p,w) \in F\) and\(\delta^*(q,w) \notin
F\), then it contradicts to the assumption that \(not\ 
exactly\ one\ of\ \delta^*(p,w)\ and\ \delta^*(q,w)\ is\ 
an\ accepting\ state\). Therefore, \(\delta^*(q,w) \in F\) is true;
the same argument applies for the case when \(\delta^*(p,w) \notin F\)
and \(\delta^*(q,w) \in F\).
\end{proof}


\begin{lem}
\label{lem:sim_nsim}
\noindent For any two states \(p\) and \(q\) in a given DFA, \(\neg (p
\sim q)\) if and only if \(p \nsim q\). 
\end{lem}

\begin{proof}
\noindent We have to prove for both directions.

\par \noindent 1) Case \(\neg (p \sim q) \Rightarrow p \nsim
q\): suppose \(p \nsim q\) is false, then there does not exist a
string \(w\) that can distinguish \(p\) and \(q\). This implies that for any string \(w\), \(w\) cannot
distinguish \(p\) and \(q\), i.e. \(\forall w.\ \neg (\delta^*(p,w) \in F \wedge \delta^*(q,w) \notin F \vee
\delta^*(p,w) \notin F \wedge \delta^*(q,w) \in F)\). By
\autoref{lem:sim_nsim1}, we have \(\forall w.\ \delta^*(p,w) \in F
\Leftrightarrow \delta^*(q,w)\) which is equivalent to \(p \sim
q\). This contradicts to the assumption that \(\neg (p \sim q)\);
therefore, \(p \nsim q\) must be true. 

\par \noindent 2) Case \(p \nsim q \Rightarrow \neg (p \sim q)\): suppose \(p \sim q\), then for any string \(w\),
\(\delta^*(p,w) \in F \Leftrightarrow \delta^*(q,w)\). Since \(p \nsim
q\), then there must exist a string \(w\)
that can distinguish \(p\) and \(q\), i.e. exactly one of
\(\delta^*(p,w)\) and \(\delta^*(q,w)\) is an accepting state. This
contradicts to the assumption that \(\delta^*(p,w) \in F
\Leftrightarrow \delta^*(q,w)\); therefore, \(p \sim q\) must be true. 
\end{proof}

\begin{thm}
\label{thm:mdfa_irreducible}
\noindent For any MDFA, it is irreducible, i.e. for any two states
\(p\) and \(q\) in the MDFA, if \(p\) is not equal to \(q\), then \(p \nsim q\). 
\end{thm}

\begin{proof}
\noindent For any two states \(\llangle p \rrangle\) and \(\llangle q
\rrangle\) in MDFA, if \(\llangle p \rrangle\) is not equal to \(\llangle q
\rrangle\), then \(p \sim q\) is false. By \autoref{lem:sim_nsim},
\(p \nsim q\) is true. 
\end{proof}

\begin{thm}
\noindent For any MDFA, it is minimal, i.e. all the states in the MDFA
are reachable and the MDFA is irreducible. 
\end{thm}

\begin{proof}
\noindent The statement follows from \autoref{thm:all_reach}
and \autoref{thm:mdfa_irreducible}. 
\end{proof}


