\chapter{Agda}
\par Agda is a dependently-typed functional programming language and a
proof assistant based on Intuitionistic Type Theory
\cite{martin1984}. The current version (Agda 2) is rewritten by Norell
\cite{norell2007} during his doctorate study at the Chalmers University of
Technology. In this chapter, we will describe the basic features of
Agda and how dependent types are employed to construct programs and
proofs. Most of the materials presented below can also be
found in the two tutorial papers \cite{bove2009} and
\cite{norell2008}. Interested readers can read the two papers in order to get
a more precise idea on how to work with Agda. We will begin by
showing how to do ordinary functional programming in Agda. 


\section{Simply Typed Functional Programming}
\par Haskell is the implementation language
of Agda and as shown below, Agda has borrowed many features from
Haskell. In the following paragraphs, we will demonstrate how to
define basic data types and functions. Let us begin with the boolean
data type. 

\paragraph{Boolean} In Haskell, the boolean type can be defined as
\(data\ Bool = True\ |\ False\). While in Agda, the syntax of the
declaration is slightly different. 
\begin{lstlisting}[mathescape=true,xleftmargin=.3\textwidth]
data Bool : Set where
  true  : Bool
  false : Bool
\end{lstlisting}

\par \mb{Bool} has two constructors: \mb{true} and \mb{false}. These two constructors
are also the elements of \mb{Bool} as they take no arguments. Furthermore, \mb{Bool} itself is a member of the type
\mb{Set}. The type of \mb{Set} is \mb{Set_1} and the type of \mb{Set_1} is
\mb{Set_2}. The type hierarchy goes on and becomes infinite. Now, let us define the negation of boolean values. 
\begin{lstlisting}[mathescape=true,xleftmargin=.3\textwidth]
not : Bool $\to$ Bool
not true  = false
not false = true
\end{lstlisting}

\par Unlike in Haskell, a type signature must be provided explicitly for every
function. Furthermore, all possible cases must be pattern matched in
the function body. For instance, the function below will be rejected
by the Agda compiler as the case \mb{(false)} is missing. 
\begin{lstlisting}[mathescape=true,xleftmargin=.3\textwidth]
not : Bool $\to$ Bool
not true  = false
\end{lstlisting}

\paragraph{Natural Number} We will define the type of natural
numbers in Peano style. 
\begin{lstlisting}[mathescape=true,xleftmargin=.3\textwidth]
data $\mathbb N$ : Set where
  zero : $\mathbb N$
  suc  : $\mathbb N$ $\to$ $\mathbb N$
\end{lstlisting} 

\par The constructor \mb{suc} represents the successor of a given
natural number. For instance, the number \mb{1} is equivalent to
(\mb{suc\ zero}). Now, let us define the addition of natural numbers recursively as follow:
\begin{lstlisting}[mathescape=true,xleftmargin=.3\textwidth]
_+_ : $\mathbb N$ $\to$ $\mathbb N$ $\to$ $\mathbb N$
zero + m = m
(suc n) + m = suc (n + m)
\end{lstlisting} 

\paragraph{Parameterised Types} In Haskell, the type of list \mb{[a]} is parameterised by the type
parameter \mb{a}. The analogous data type in Agda is defined as follow:
\begin{lstlisting}[mathescape=true,xleftmargin=.3\textwidth]
data List (A : Set) : Set where
  []   : List A
  _::_ : A $\to$ List A $\to$ List A
\end{lstlisting} 

\par Now, let us try to define a function that will return the first
element of a given list. 
\begin{lstlisting}[mathescape=true,xleftmargin=.3\textwidth]
head : {A : Set} $\to$ List A $\to$ A
head [] = {!!}
head (x :: xs) = x
\end{lstlisting} 

\par What should be returned for case \mb{[\ ]}? In Haskell, the \mb{[\
  ]} case can simply be ignored and an error will be produced by the
compiler. However, as we have mentioned
before, all possible cases must be pattern matched in an Agda function. One
possible workaround is to return \mb{nothing} -- an element of the
\mb{Maybe} type. Another solution is
to constrain the argument using dependent types such that the input
list will always have at least one element. 


\section{Dependent Types}
\par A dependent type is a type that depends on values of other
types. For example, \mb{A^n} is a vector that contains \mb{n} elements
of \mb{A}. It is impossible to declare these kinds of types in
simply-typed systems like Haskell\footnotemark and Ocaml. Now, let
us look at how it is declared in Agda.
\footnotetext{Haskell does not support dependent types by
  its own. However, there are several APIs that simulate
  dependent types, for example, Ivor \cite{ivor2016} and
  GADT.}
\begin{lstlisting}[mathescape=true,xleftmargin=.25\textwidth]
data Vec (A : Set) : $\mathbb N$ $\to$ Set where
  []   : Vec A zero
  _::_ : $\forall$ {n} $\to$ A $\to$ Vec A n $\to$ Vec A (suc n)
\end{lstlisting} 

\par In the type signature, (\mb{\mathbb N \to Set}) means
that \mb{Vec} takes a number \mb{n} from \mb{\mathbb N} and produces a
type that depends on \mb{n}. The inductive family \mb{Vec} will
produce different types with different numbers. For example, (\mb{Vec\ A\ zero}) is
the type of empty vectors and (\mb{Vec\ A\ 10}) is another vector type with length ten. 

\par Dependent types allow us to be more
expressive and precise over type declaration. Let us define the
\mb{head} function for \mb{Vec}. 
\begin{lstlisting}[mathescape=true,xleftmargin=.25\textwidth]
head : {A : Set}{n : $\mathbb N$} $\to$ Vec A (suc n) $\to$ A
head (x :: xs) = x 
\end{lstlisting} 

\par Only the (\mb{x :: xs}) case needs to be pattern matched because
an element of the vector type (\(Vec\ A\ (suc\ n)\)) must be in the form of
(\mb{x :: xs}). Apart from vectors, a type of binary
search tree can also be declared in which any tree of this type is guaranteed to be
sorted. Interested readers can take a look at 
Chapter 6 in \cite{bove2009}. Furthermore, dependent types also allow
us to encode predicate logic and program specifications as
types. These two applications will be described in later part after we
have discussed the idea of propositions as types. 


\section{Propositions as Types}
\par In the 1930s, Curry identified the
correspondence between propositions in propositional logic and types
\cite{curry1934}. After that, in the 1960s, de Bruijn and Howard extended
Curry's correspondence to predicate logic by introducing dependent
types \cite{bruijn1968, howard1969}. Later on, Martin-L\"of published
his work, Intuitionistic Type Theory \cite{martin1984}, which turned the correspondence into a new
foundational system for constructive mathematics. 

\par In the paragraphs below, we will show how the correspondence is
formalised in Agda. Note that Intuitionistic Type
Theory is based on constructive logic but not classical logic and there
is a fundamental difference between them. Interested readers can take a look at
\cite{avigad2000}. Now, we will begin by showing how propositional
logic is formalised in Agda. 

\subsection{Propositional Logic} 
\par In general, Curry's correspondence
states that a proposition can be interpreted as a set of its proofs. A
proposition is true if and only if its set of proofs is inhabited,
i.e. there is at least one element in the set; it is false if and only
if its set of proofs is empty. Let us begin with the formalisation of
\(Truth\) -- the proposition that is always true. 

\paragraph{Truth} For a proposition to be always true, its
corresponding type must have at least one element. 
\begin{lstlisting}[mathescape=true,xleftmargin=.3\textwidth]
data $\top$ : Set where
  tt : $\top$
\end{lstlisting} 

\paragraph{Falsehood} The proposition that is always
false corresponds to a type having no elements at all. 
\begin{lstlisting}[mathescape=true,xleftmargin=.3\textwidth]
data $\bot$ : Set where
\end{lstlisting} 

\paragraph{Conjunction} Suppose \mb{A} and \mb{B} are propositions, then a
proof of their conjunction, \mb{A \wedge B}, should contain both a proof of \mb{A} and a proof
of \mb{B}. In Type Theory, it corresponds to the product type. 
\begin{lstlisting}[mathescape=true,xleftmargin=.3\textwidth]
data _$\times$_ (A B : Set) : Set where
  _,_ : A $\to$ B $\to$ A $\times$ B
\end{lstlisting} 

\par The above construction resembles the introduction rule of
conjunction. The elimination rules are formalised as follow:
\begin{lstlisting}[mathescape=true,xleftmargin=.3\textwidth]
fst : {A B : Set} $\to$ A $\times$ B $\to$ A
fst (a , b) = a

snd : {A B : Set} $\to$ A $\times$ B $\to$ B
snd (a , b) = b
\end{lstlisting} 

\paragraph{Disjunction} Suppose \mb{A} and \mb{B} are propositions, then a
proof of their disjunction, \mb{A \vee B}, should contain either a proof of \mb{A} or a
proof of \mb{B}. In Type Theory, it is represented by the sum type. 
\begin{lstlisting}[mathescape=true,xleftmargin=.3\textwidth]
data _$\uplus$_ (A B : Set) : Set where
  inj$_1$ : A $\to$ A $\uplus$ B
  inj$_2$ : B $\to$ A $\uplus$ B
\end{lstlisting} 

\par The elimination rule of disjunction is defined as follow: 
\begin{lstlisting}[mathescape=true,xleftmargin=.3\textwidth]
$\uplus \hyphen$elim : {A B C : Set} 
         $\to$ A $\uplus$ B 
         $\to$ (A $\to$ C) 
         $\to$ (B $\to$ C) 
         $\to$ C
$\uplus \hyphen$elim (inj$_1$ a) f g = f a
$\uplus \hyphen$elim (inj$_2$ b) f g = g b
\end{lstlisting} 

\paragraph{Negation} Suppose \mb{A} is a proposition, then its negation is
defined as a function that transforms any arbitrary proof of \mb{A} into
the falsehood (\mb{\bot}). 
\begin{lstlisting}[mathescape=true,xleftmargin=.3\textwidth]
$\neg$ : Set $\to$ Set
$\neg$ A = A $\to$ $\bot$
\end{lstlisting} 


\paragraph{Implication} We say that \mb{A} implies \mb{B} if and only
if every proof of \mb{A} can be transformed into a proof of \mb{B}. In Type
Theory, it corresponds to a function from \mb{A} to \mb{B}, i.e. \mb{A \to
B}. 

\paragraph{Equivalence} Two propositions \mb{A} and
\mb{B} are equivalent if and only if \mb{A} implies \mb{B} and \mb{B} implies
\mb{A}. It can be considered as a conjunction of the two implications.
\begin{lstlisting}[mathescape=true,xleftmargin=.3\textwidth]
_$\iff$_ : Set $\to$ Set $\to$ Set
A $\iff$ B = (A $\to$ B) $\times$ (B $\to$ A)
\end{lstlisting} 

\par Now, by using the above constructions, we can formalise
 theorems in propositional logic. For example, we can prove that if \mb{P} implies \mb{Q} and
\mb{Q} implies \mb{R}, then \mb{P} implies \mb{R}. The corresponding
proof in Agda is as follow:
\begin{lstlisting}[mathescape=true,xleftmargin=.3\textwidth]
prop-lem : {P Q : Set} 
           $\to$ (P $\to$ Q) 
           $\to$ (Q $\to$ R) 
           $\to$ (P $\to$ R)
prop-lem f g = $\lambda$ p $\to$ g (f p)
\end{lstlisting} 

\par By completing the function, we have provided an element
to the type \((P \to Q) \to (Q \to R) \to (P \to R)\) and thus we have
also proved the theorem to be true. 


\subsection{Predicate Logic} 
\par We will now move on to predicate logic and
introduce the universal (\mb{\forall}) and existential (\mb{\exists})
quantifiers. Suppose \(A\) is a set, then a predicate on \(A\) corresponds to a dependent type in the
form of \mb{A \to Set}. For example, we can
define the predicates of even numbers and odd numbers inductively as follow:
\begin{lstlisting}[mathescape=true,xleftmargin=.25\textwidth]
mutual
  data _isEven : $\mathbb N$ $\to$ Set where
    base : zero isEven
    even : $\forall$ n $\to$ n isOdd $\to$ (suc n) isEven

  data _isOdd : $\mathbb N$ $\to$ Set where
    odd : $\forall$ n $\to$ n isEven $\to$ (suc n) isOdd
\end{lstlisting} 

\paragraph{Universal Quantifier} The interpretation of the universal
quantifier is similar to that of implication. In order for \mb{\forall x\in A.\ B(x)} to be true, every
proof \mb{x} of \mb{A} must be transformed into a proof of the predicate
\mb{B[a:=x]}. In Type Theory, it is represented by the function \mb{(x :
A) \to B\ x}. For example, we can prove by induction that for every natural
number, it is either even or odd.
\begin{lstlisting}[mathescape=true,xleftmargin=.25\textwidth]
lem$_1$ : $\forall$ n $\to$ n isEven $\uplus$ n isOdd
lem$_1$ zero = inj$_1$ base
lem$_1$ (suc n) with lem$_1$ n
... | inj$_1$ nIsEven = inj$_2$ (even n nIsEven)
... | inj$_2$ nIsOdd = inj$_1$ (odd n nIsOdd)
\end{lstlisting} 

\paragraph{Existential Quantifier} The interpretation of the
existential quantifier is similar to that of conjunction. In order for
\mb{\exists x\in A.\ B(x)} to be true, a proof \mb{x} of \mb{A} and a proof of the predicate
\mb{B[a:=x]} must be provided. In Type Theory, it is represented by the generalised
product type \mb{\Sigma}. 
\begin{lstlisting}[mathescape=true,xleftmargin=.25\textwidth]
data $\Sigma$ (A : Set) (B : A $\to$ Set) : where
  _,_ : (a : A) $\to$ B a $\to$ $\Sigma$ A B
\end{lstlisting}

\par For simplicity, we will change the syntax of \mb{\Sigma} type to
\mb{\exists [ x \in A ]\ B\ x}. As an example, let us prove that
there exists an even number. 
\begin{lstlisting}[mathescape=true,xleftmargin=.25\textwidth]
lem$_2$ : $\exists$[ n $\in$ $\mathbb N$ ] (n isEven)
lem$_2$ = zero , base
\end{lstlisting}


\subsection{Decidability} 
\par A proposition is decidable if and only if there
exists an algorithm that can decide whether the proposition is true or false. It is
defined in Agda as follow: 
\begin{lstlisting}[mathescape=true,xleftmargin=.25\textwidth]
data Dec (A : Set) : Set where
  yes : A $\to$ Dec A
  no  : $\neg$ A $\to$ Dec A
\end{lstlisting}

\par For example, we can prove that the predicate of even numbers is
decidable. Interested readers can try and complete the following proofs. 
\begin{lstlisting}[mathescape=true,xleftmargin=.25\textwidth]
postulate lem$_3$ : $\forall$ n $\to$ $\neg$ (n isEven $\times$ n isOdd)

postulate lem$_4$ : $\forall$ n $\to$ n isEven $\to$ $\neg$ ((suc n) isEven)

postulate lem$_5$ : $\forall$ n $\to$ $\neg$ (n isEven) $\to$ (suc n) isEven

even-dec : $\forall$ n $\to$ Dec (n isEven)
even-dec zero = yes base
even-dec (suc n) with even-dec n
... | yes nIsEven = no (lem$_4$ n nIsEven)
... | no $\neg$nIsEven = yes (lem$_5$ n $\neg$nIsEven)
\end{lstlisting} 


\subsection{Propositional Equality} 
\par One of the important features of Type Theory is
to encode the equality relation of propositions as types. The
equality relation is interpreted as follow:
\begin{lstlisting}[mathescape=true,xleftmargin=.25\textwidth]
data _$\equiv$_ {A : Set} (x : A) : A $\to$ Set where
  refl : x $\equiv$ x
\end{lstlisting}

\par This states that for any \mb{x} in \mb{A}, \mb{refl} is an
element of the type \mb{x \equiv x}. More generally, \mb{refl} is a
proof of \mb{x \equiv x'} provided that \mb{x} and \mb{x'} are the same
after normalisation. For example, we can prove that \mb{suc\ (suc\ zero) \equiv 1 + 1} as follow:
\begin{lstlisting}[mathescape=true,xleftmargin=.25\textwidth]
lem$_3$ : suc (suc zero) $\equiv$ (1 + 1)
lem$_3$ = refl
\end{lstlisting}

\par We can put \mb{refl} in the proof only because both
\mb{suc\ (suc\ zero)} and \mb{1 + 1} have the same form after
normalisation. Now, let us define the elimination rule of
equality. The rule should allow us to substitute equivalence objects
into any proposition. 
\begin{lstlisting}[mathescape=true,xleftmargin=.05\textwidth]
subst : {A : Set}{x y : A} $\to$ (P : A $\to$ Set) $\to$ x $\equiv$ y $\to$ P x $\to$ P y
subst P refl p = p 
\end{lstlisting}

\par We can also prove the congruency of equality.
\begin{lstlisting}[mathescape=true,xleftmargin=.05\textwidth]
cong : {A B : Set}{x y : A} $\to$ (f : A $\to$ B) $\to$ x $\equiv$ y $\to$ f x $\equiv$ f y
cong f refl = refl
\end{lstlisting}


\section{Program Specifications as Types}
\par As we have mentioned before, dependent types also allow us to encode program
specifications within the same platform. In order to demonstrate the
idea, we will give an example on the insertion function of sorted
lists. Let us begin by defining a predicate
of sorted list (in ascending order). For simplicity, we will only consider the list of natural
numbers. 
\begin{lstlisting}[mathescape=true,xleftmargin=.25\textwidth]
All$\hyphen$lt : $\mathbb N$ $\to$ List $\mathbb N$ $\to$ Set
All$\hyphen$lt n [] = $\top$
All$\hyphen$lt n (x :: xs) = n $\leq$ x $\times$ All$\hyphen$lt n xs

Sorted$\hyphen$ASC : List $\mathbb N$ $\to$ Set
Sorted$\hyphen$ASC [] = $\top$
Sorted$\hyphen$ASC (x :: xs) = All$\hyphen$lt x xs $\times$ Sorted$\hyphen$ASC xs
\end{lstlisting}

\par Note that \(All\hyphen lt\) defines the condition where a given
number is smaller than all the numbers inside a given list. Now, let
us define an insertion function that takes a natural number and a list as the arguments and returns a list of
natural numbers. The insertion function is designed in a way that if the
input list is already sorted, then the output list will also be sorted. 
\begin{lstlisting}[mathescape=true,xleftmargin=.25\textwidth]
insert : $\mathbb N$ $\to$ List $\mathbb N$ $\to$ List $\mathbb N$
insert n [] = n :: []
insert n (x :: xs) with n $\leq$? x
... | yes _ = n :: (x :: xs)
... | no  _ = x :: insert n xs
\end{lstlisting}

\par Note that \mb{\leq ?} has the type \mb{\forall\ n\ m \to Dec\ (n \leq
  m)}. It is a proof of the decidability of \mb{\leq} and it can also be used to determine whether
a given number \mb{n} is less than or equal to another number
\mb{m}. Now, let us encode the specification of the insertion function
as follow: 
\begin{lstlisting}[mathescape=true,xleftmargin=.25\textwidth]
insert$\hyphen$sorted : $\forall$ {n} {as} 
                $\to$ Sorted$\hyphen$ASC as 
                $\to$ Sorted$\hyphen$ASC (insert n as)
\end{lstlisting}

\par In the type signature, the argument \((Sorted\hyphen ASC\ as)\) corresponds to
the pre-condition while the argument 
\((Sorted\hyphen ASC\ (insert\ n\ as))\) corresponds to the
post-condition. Once we have completed the
function, we will have also proved the specification to be
true. Readers are recommended to finish the proof. 