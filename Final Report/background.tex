\section{Agda}
\par Agda is a dependently-typed functional programming language and a
proof assistant based on Intuitionistic Type Theory
\cite{martin1984}. The current version (Agda 2) is rewritten by Norell
\cite{norell2007} during his doctorate study at the Chalmers University of
Technology. In this section, we will describe the basic features of
Agda and how dependent types are employed to construct programs and
proofs. Most of the materials presented below can also be
found in the two tutorial papers \cite{bove2009} and
\cite{norell2009}. Interested readers can read the two papers in order to get
a more precise idea on how to work with Agda. Now, we will begin by
showing how to do ordinary functional programming in Agda. 


\subsection{Simply Typed Functional Programming}
\par Haskell is the implementation language
of Agda, as shown below, Agda has borrowed many features from
Haskell. In the following paragraphs, we will demonstrate how to
define basic data types and functions. 

\paragraph{Boolean} We first declare the type of Boolean values in Agda.  
\begin{lstlisting}[mathescape=true,xleftmargin=.3\textwidth]
data Bool : Set where
  true  : Bool
  false : Bool
\end{lstlisting}

\par \mb{Bool} has two
constructors: \mb{true} and \mb{false}. These two constructors
are also elements of \mb{Bool} as they take no arguments. On
the other hand, \mb{Bool} itself is a member of the type
\mb{Set}. The type of \mb{Set} is \mb{Set_1} and the type of \mb{Set_1} is
\mb{Set_2}. The type hierarchy goes on and becomes infinite. Now, let
us define the negation of Boolean values. 
\begin{lstlisting}[mathescape=true,xleftmargin=.3\textwidth]
not : Bool $\to$ Bool
not true  = false
not false = true
\end{lstlisting}

\par Unlike in Haskell, a type signature must be provided explicitly for every
function and all possible cases must be included in the function body
while pattern matching. For instance, the function below will be rejected
by the Agda compiler as the case \mb{(not\ false)} is missing. 
\begin{lstlisting}[mathescape=true,xleftmargin=.3\textwidth]
not : Bool $\to$ Bool
not true  = false
\end{lstlisting}

\paragraph{Natural Number} Now, let us declare the type of natural
numbers in Peano style. 
\begin{lstlisting}[mathescape=true,xleftmargin=.3\textwidth]
data $\mathbb N$ : Set where
  zero : $\mathbb N$
  suc  : $\mathbb N$ $\to$ $\mathbb N$
\end{lstlisting} 

\par The constructor \mb{suc} represents the successor of a given
natural number. For instance, the number \mb{1} is equivalent to
(\mb{suc\ zero}). Now, let us define the addition of natural numbers recursively as follow:
\begin{lstlisting}[mathescape=true,xleftmargin=.3\textwidth]
_+_ : $\mathbb N$ $\to$ $\mathbb N$ $\to$ $\mathbb N$: Set where
zero + m = m
(suc n) + m = suc (n + m)
\end{lstlisting} 

\paragraph{Parameterised Types} In Haskell, the type of list \mb{[a]} is parameterised by the type
parameter \mb{a}. The analogous data type in Agda is defined as follow:
\begin{lstlisting}[mathescape=true,xleftmargin=.3\textwidth]
data List (A : Set) : Set where
  []   : List A
  _::_ : A $\to$ List A $\to$ List A
\end{lstlisting} 

\par Let us try to define a function which takes a
list as the argument and returns the first element of the list. 
\begin{lstlisting}[mathescape=true,xleftmargin=.3\textwidth]
head : {A : Set} $\to$ List A $\to$ A
head [] = {!!}
head (x :: xs) = x
\end{lstlisting} 

\par What should be returned in case \mb{[\ ]}? In Haskell, the \mb{[\
  ]} case can simply be skipped and an error will be produced by the
compiler. However, as we have mentioned
before, the function body must contains all the possible cases. One
possible workaround is to return \mb{nothing} of the \mb{Maybe} type
for case \mb{[\ ]}. Another solution is
to constrain the arguments using dependent types such that the input list will always have at least one element. 


\subsection{Dependent Types}
\par A dependent type is a type that depends on values of other
types. For example, \mb{A^n} is a vector that contains \mb{n} elements
of \mb{A}. These kind of types is not possible
to be declared in simply-typed systems like Haskell\footnotemark and Ocaml. Now, let
us look at how it is declared in Agda.
\footnotetext{Haskell itself does not support dependent types by
  its own. However, there are several APIs in Haskell that simulates
  dependent types, for example, Ivor \cite{ivor2016} and
  GADT.}
\begin{lstlisting}[mathescape=true,xleftmargin=.3\textwidth]
data Vec (A : Set) : $\mathbb N$ $\to$ Set where
  []   : Vec A zero
  _::_ : $\forall$ {n} $\to$ A $\to$ Vec A n $\to$ Vec A (suc n)
\end{lstlisting} 

\par In the type signature, (\mb{A : Set}) is the type
parameter while \mb{\mathbb N \to Set} means
that \mb{Vec} takes a number \mb{n} from \mb{\mathbb N} and produces a
type that depends on \mb{n}. Different types will be produced by giving different natural
numbers to the inductive family \mb{Vec}. For example, \mb{Vec\ A\ zero} is
the type of empty vectors and \mb{Vec\ A\ 10} is another vector type with length ten. 

\par Dependent types allow us to be more
expressive and precise over type declaration. Let us declare the
\mb{head} function for \mb{Vec}. 
\begin{lstlisting}[mathescape=true,xleftmargin=.3\textwidth]
head : {A : Set}{n : $\mathbb N$} $\to$ Vec A (suc n) $\to$ A
head (x :: xs) = x 
\end{lstlisting} 

\par Only the (\mb{x :: xs}) case needs to be pattern matched because
the type \mb{Vec\ A\ (suc\ n)} ensures that the argument will never
be \mb{[\ ]}. Apart from vectors, a type of binary
search tree can also be declared in which any tree of this type is guaranteed to be
sorted. However, this is not our major concern and thus we will not be
looking into it. Interested readers can take a look at 
Section 6 in \cite{bove2009}. Furthermore, dependent types also allow
us to encode predicate logic and program specifications as
types. These two applications will be describe in later part and now,
we will first discuss the idea of propositions as types. 


\subsection{Propositions as Types}
\par In the 1930s, Curry identified the
correspondence between propositions in propositional logic and types
\cite{curry1934}. After that, in the 1960s, de Bruijn and Howard extended
Curry's correspondence to predicate logic by introducing dependent
types \cite{bruijn1968, howard1969}. Later on, Martin-L\"of published
his work, Intuitionistic Type Theory \cite{martin1984}, which turned the correspondence into a new
foundational system for constructive mathematics. 

\par In the paragraphs below, we will show how the correspondence is
formalised in Agda. Note that Intuitionistic Type
Theory is based on constructive logic but not classical logic and there
is a fundamental difference between them. Interested readers can take a look at
\cite{avigad2000}. Now, we will begin with propositional logic. 

\subsubsection{Propositional Logic} 
\par In general, Curry's correspondence
states that a proposition can be interpreted as a set of its proofs. A
proposition is true if and only if its set of proofs is inhabited,
i.e. there is at least one element in the set; it is false if and only
if its set of proofs is empty. 

\paragraph{Truth} For a proposition to be always true, its
corresponding type must have at least one element. 
\begin{lstlisting}[mathescape=true,xleftmargin=.3\textwidth]
data $\top$ : Set where
  tt : $\top$
\end{lstlisting} 

\paragraph{Falsehood} The proposition that is always
false corresponds to a type having no elements at all. 
\begin{lstlisting}[mathescape=true,xleftmargin=.3\textwidth]
data $\bot$ : Set where
\end{lstlisting} 

\paragraph{Conjunction} Suppose \mb{A} and \mb{B} are propositions, then the
proofs of their conjunction \mb{A \wedge B} should contain both a proof of \mb{A} and a proof
of \mb{B}. In Type Theory, it corresponds
to the product type. 
\begin{lstlisting}[mathescape=true,xleftmargin=.3\textwidth]
data _$\times$_ (A B : Set) : Set where
  _,_ : A $\to$ B $\to$ A $\times$ B
\end{lstlisting} 

\par The above construction resembles the introduction rule of
conjunction while the elimination rules are formalised as follow:
\begin{lstlisting}[mathescape=true,xleftmargin=.3\textwidth]
fst : {A B : Set} $\to$ A $\times$ B $\to$ A
fst (a , b) = a

snd : {A B : Set} $\to$ A $\times$ B $\to$ B
snd (a , b) = b
\end{lstlisting} 


\paragraph{Disjunction} Suppose \mb{A} and \mb{B} are propositions, then the
proofs of their disjunction \mb{A \vee B} should contains either a proof of \mb{A} or a
proof of \mb{B}. In Type Theory, it is represented by the sum type. 
\begin{lstlisting}[mathescape=true,xleftmargin=.3\textwidth]
data _$\uplus$_ (A B : Set) : Set where
  inj$_1$ : A $\to$ A $\uplus$ B
  inj$_2$ : B $\to$ A $\uplus$ B
\end{lstlisting} 

\par The elimination rule of disjunction is defined as follow: 
\begin{lstlisting}[mathescape=true,xleftmargin=.3\textwidth]
$\uplus \hyphen$elim : {A B C : Set} 
         $\to$ A $\uplus$ B 
         $\to$ (A $\to$ C) 
         $\to$ (B $\to$ C) 
         $\to$ C
$\uplus \hyphen$elim (inj$_1$ a) f g = f a
$\uplus \hyphen$elim (inj$_2$ b) f g = g b
\end{lstlisting} 

\paragraph{Negation} Suppose \mb{A} is a proposition, then its negation is
defined as a function that transforms any arbitrary proof of \mb{A} into
the falsehood (\mb{\bot}). 
\begin{lstlisting}[mathescape=true,xleftmargin=.3\textwidth]
$\neg$ : Set $\to$ Set
$\neg$ A = A $\to$ $\bot$
\end{lstlisting} 


\paragraph{Implication} We say that \mb{A} implies \mb{B} if and only
if every proof of \mb{A} can be transformed into a proof of \mb{B}. In Type
Theory, it corresponds to a function from \mb{A} to \mb{B}, i.e. \mb{A \to
B}. 

\paragraph{Equivalence} Two propositions \mb{A} and
\mb{B} are equivalent if and only if \mb{A} implies \mb{B} and \mb{B} implies
\mb{A}. It can be considered as a conjunction of the two implications.
\begin{lstlisting}[mathescape=true,xleftmargin=.3\textwidth]
_$\iff$_ : Set $\to$ Set $\to$ Set
A $\iff$ B = (A $\to$ B) $\times$ (B $\to$ A)
\end{lstlisting} 

\paragraph{} Now, by using the above constructions, we can formalise
 theorems of propositional logic in Agda. For example, we can prove that if \mb{P} implies \mb{Q} and
\mb{Q} implies \mb{R}, then \mb{P} implies \mb{R}. 
\begin{lstlisting}[mathescape=true,xleftmargin=.3\textwidth]
prop-lem : {P Q : Set} 
           $\to$ (P $\to$ Q) 
           $\to$ (Q $\to$ R) 
           $\to$ (P $\to$ R)
prop-lem f g = $\lambda$ p $\to$ g (f p)
\end{lstlisting} 

\par By completing the function, we have provided an element
to the type \mb{(P \to Q) \to (Q \to R) \to (P \to R)} and thus, we have
also proved the theorem to be true. 


\subsubsection{Predicate Logic} 
\par We will now move on to predicate logic and
introduce the universal (\mb{\forall}) and existential (\mb{\exists})
quantifiers. A predicate is represented by a dependent type in the
form of \mb{A \to Set}. For example, we can
define the predicate of even numbers and odd numbers inductively as follow:
\begin{lstlisting}[mathescape=true,xleftmargin=.3\textwidth]
mutual
  data _isEven : $\mathbb N$ $\to$ Set where
    base : zero isEven
    step : $\forall$ n $\to$ n isOdd $\to$ (suc n) isEven

  data _isOdd : $\mathbb N$ $\to$ Set where
    step : $\forall$ n $\to$ n isEven $\to$ (suc n) isOdd
\end{lstlisting} 

\paragraph{Universal Quantifier} The interpretation of the universal quantifier is similar to
implication. In order for \mb{\forall x\in A.\ B(x)} to be true, every
proof \mb{(a)} of \mb{A} must be transformed into a proof of the predicate
\mb{B[x:=a]}. In Type Theory, it is represented by the function \mb{(x :
A) \to B\ x}. For example, we can prove by induction that for every natural
number, it is either even or odd.
\begin{lstlisting}[mathescape=true,xleftmargin=.3\textwidth]
lem$_1$ : $\forall$ n $\to$ n isEven $\uplus$ n isOdd
lem$_1$ zero = inj$_1$ base
lem$_1$ (suc n) with lem$_1$ n
... | inj$_1$ nIsEven = inj$_2$ (step n nIsEven)
... | inj$_2$ nIsOdd = inj$_1$ (step n nIsOdd)
\end{lstlisting} 

\paragraph{Existential Quantifier} The interpretation of the
existential quantifier is similar to conjunction. In order for
\mb{\exists x\in A.\ B(x)} to be true, a proof
\mb{(a)} of \mb{A} and a proof \mb{(p)} of the predicate
\mb{B[x:=a]} must be provided. In Type Theory, it is represented by the generalised
product type \mb{\Sigma}. 
\begin{lstlisting}[mathescape=true,xleftmargin=.3\textwidth]
data $\Sigma$ (A : Set) (B : A $\to$ Set) : where
  _,_ : (a : A) $\to$ B a $\to$ $\Sigma$ A B
\end{lstlisting}
\par For simplicity, we will change the syntax of \mb{\Sigma} to
\mb{\exists [ x \in A ]\ B}. As an example, we can prove that
there exists a natural number which is even. 
\begin{lstlisting}[mathescape=true,xleftmargin=.3\textwidth]
lem$_2$ : $\exists$[ n $\in$ $\mathbb N$ ] (n isEven)
lem$_2$ = zero , base
\end{lstlisting}


\subsubsection{Decidability} 
\par A proposition \mb{P} is decidable if and only if there
exists an algorithm that can decide whether it is true or false. It is
defined as follow: 
\begin{lstlisting}[mathescape=true,xleftmargin=.3\textwidth]
data Dec (A : Set) : Set where
  yes : A $\to$ Dec A
  no  : $\neg$ A $\to$ Dec A
\end{lstlisting}

\par For example, we can prove that the predicate of even numbers is
decidable. Interested readers can try and complete the proofs. 
\begin{lstlisting}[mathescape=true,xleftmargin=.3\textwidth]
lem$_3$ : $\forall$ n $\to$ $\neg$ (n isEven $\times$ n isOdd)
lem$_3$ = ?

lem$_4$ : $\forall$ n $\to$ n isEven $\to$ $\neg$ ((suc n) isEven)
lem$_4$ = ?

lem$_5$ : $\forall$ n $\to$ $\neg$ (n isEven) $\to$ (suc n) isEven
lem$_5$ = ?

even-dec : $\forall$ n $\to$ Dec (n isEven)
even-dec zero = yes base
even-dec (suc n) with even-dec n
... | yes nIsEven = no (lem$_4$ n nIsEven)
... | no $\neg$nIsEven = yes (lem$_5$ n $\neg$nIsEven)
\end{lstlisting}


\subsubsection{Propositional Equality} 
\par One important feature of Type Theory is
that the equality of propositions can also be defined as types. The
equality relation is interpreted as follow:
\begin{lstlisting}[mathescape=true,xleftmargin=.3\textwidth]
data _$\equiv$_ {A : Set} (x : A) : A $\to$ Set where
  refl : x $\equiv$ x
\end{lstlisting}

\par This states that for any \mb{x} in \mb{A}, \mb{refl} is an
element of the type \mb{x \equiv x}. More generally, \mb{refl} is a
proof of \mb{x \equiv x'} provided that \mb{x} and \mb{x'} is the same
after normalisation. For example, we can prove that \mb{\exists n\in
\mathbb N .\ n = 1 + 1} as follow:
\begin{lstlisting}[mathescape=true,xleftmargin=.3\textwidth]
lem$_3$ : $\exists$[ n $\in$ $\mathbb N$ ] n $\equiv$ (1 + 1)
lem$_3$ = suc (suc zero) , refl
\end{lstlisting}

\par We can put \mb{refl} in the proof only because both
\mb{suc\ (suc\ zero)} and \mb{1 + 1} have the same form after
normalisation. Now, let us define the elimination rule of
equality. The rule should allow us to substitute equivalence objects
into any proposition. 
\begin{lstlisting}[mathescape=true,xleftmargin=.3\textwidth]
subst : {A : Set}{x y : A} $\to$ (P : A $\to$ Set) $\to$ x $\equiv$ y $\to$ P x $\to$ P y
subst P refl p = p 
\end{lstlisting}

\par We can also prove the congruency of equality.
\begin{lstlisting}[mathescape=true,xleftmargin=.3\textwidth]
cong : {A B : Set}{x y : A} $\to$ (f : A $\to$ B) $\to$ x $\equiv$ y $\to$ f x $\equiv$ f y
cong f refl = refl
\end{lstlisting}


\subsection{Program Specifications as Types}
\par As we have mentioned before, dependent types also allow us to encode program
specifications within the same platform. In order to demonstrate the
idea, we will use the insertion function of sorted lists as an
example. Let us begin by defining a predicate
of sorted list (in ascending order). 
\begin{lstlisting}[mathescape=true,xleftmargin=.3\textwidth]
All$\hyphen$lt : $\mathbb N$ $\to$ List $\mathbb N$ $\to$ Set
All$\hyphen$lt n [] = $\top$
All$\hyphen$lt n (x :: xs) = n $\leq$ x $\times$ All$\hyphen$lt n xs

Sorted$\hyphen$ASC : List $\mathbb N$ $\to$ Set
Sorted$\hyphen$ASC [] = $\top$
Sorted$\hyphen$ASC (x :: xs) = All$\hyphen$lt x xs $\times$ Sorted$\hyphen$ASC xs
\end{lstlisting}

\par For simplicity, only the list of natural numbers is
considered. Note that \(All\hyphen lt\) defines the condition where a given
number is smaller than all the numbers inside a given list. Now, let
us define an insertion function that takes a natural number and a list as the arguments and returns a list of
natural numbers. The insertion function is designed in a way that if the
input list is already sorted, then the output list will also be sorted. 
\begin{lstlisting}[mathescape=true,xleftmargin=.3\textwidth]
insert : $\mathbb N$ $\to$ List $\mathbb N$ $\to$ List $\mathbb N$
insert n [] = n :: []
insert n (x :: xs) with n $\leq$? x
... | yes _ = n :: (x :: xs)
... | no  _ = x :: insert n xs
\end{lstlisting}

\par Note that \mb{\_\leq ?\_} has the type \mb{\forall\ n\ m \to Dec\ (n \leq
  m)}. It is a proof of the decidability of \mb{\_\leq\_} and it can also be used to determine whether
a given number \mb{n} is less than or equal to another number
\mb{m}. Now, let us encode the specification of the insertion function
as follow: 
\begin{lstlisting}[mathescape=true,xleftmargin=.3\textwidth]
insert$\hyphen$sorted : $\forall$ {n} {as} 
                $\to$ Sorted$\hyphen$ASC as 
                $\to$ Sorted$\hyphen$ASC (insert n as)
\end{lstlisting}

\par In the type signature, \((Sorted\hyphen ASC\ as)\) corresponds to the pre-condition and
\((Sorted\hyphen ASC\ (insert\ n\ as))\) corresponds to the
post-condition. Once we have completed the
function, we will also have proved the specification to be
true. Interested readers are recommended to finished the proof. 



\newpage
\section{Related Work}

\subsection{Regular Expressions in Agda}
\par Agular and Mannaa published a
similar work \cite{agular2009} in 2009. They constructed a decider for
regular expressions which can determine whether
a given string is accepted by a given regular expression. The decider was based on the calculation of the derivation of a regular
expression which only needs to convert the regular expression into
part of an automaton. Their decider was implemented using the \mb{Maybe} type as follow:
\begin{lstlisting}[mathescape=true,xleftmargin=.3\textwidth]
accept : (re : RegExp) $\to$ (as : List carrier) $\to$ Maybe (as $\in\smile [\![ re ]\!]$)
\end{lstlisting}
\par When a string is accepted by the regular expression, i.e. \mb{w
  \in L(e)}, the decider will return its proof. However it fails to generate a
proof for the opposite case, i.e. \mb{w \notin L(e)}. As they
explained in the paper, it is not possible without converting the regular expression into
the entire finite automaton. 


\subsection{Certified Parsing of Regular Languages in Agda}
\par While in 2013, Firsov and Uustalu also published another related
research paper \cite{firsov2013}. They translated regular expressions
into NFA and proved that their accepting languages are
equal. Unlike Agular and Mannaa's decider, Firsov and Uustalu's
algorithm could generate proofs for both cases. In their definition of NFA, the set of states
\mb{Q} and its subsets are represented as vectors while the transition function
\mb{\delta} takes an alphabet as the argument and returns a matrix
representation of the transition table. 
\begin{lstlisting}[mathescape=true,xleftmargin=.3\textwidth]
record NFA : Set where
  field
    |Q| : $\mathbb N$
    $\delta$    : $\Sigma$ $\to$ |Q| $\ast$ |Q|
    I   : 1 $\ast$ |Q|
    F   : |Q| $\ast$ 1
\end{lstlisting}

\par Note that \mb{\_\ast\_} is an inductive family that takes two
natural numbers \mb{n} and \mb{m} and
produces a matrix type \mb{n \times m}. This representation allows us to
iterate the set easily but it looks unnatural compare to the actual 
mathematical definition of NFA. 