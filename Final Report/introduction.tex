\section{Introduction}
\par This project aims to study the feasibility of formalising
Automata Theory \cite{aho1972} in Type Theory \cite{martin1984}. 
... [ small intro to type theory \(\to\) type theory and proof assistant and
dependent types \(\to\) Agda will be used ] 

\par Automata Theory is an extensive work; therefore, it will be unrealistic to
include all the materials under the time constraints. Accordingly,
this project will only focus on the theorems and
proofs that are related to the translation of regular expressions
to finite automata. In addition, this project also serves as an
example of how complex and non-trivial proofs are formalised. 

\par Our Agda formalisation consists of two components: 1) the
translation of regular expressions to DFA and 2)
the correctness proofs of the translation. At this stage, we are only
interested in the correctness of the translation but not the
efficiency of the algorithms. 


\subsection{Motivation}
\par My motivation on this project is to learn and apply
dependent types in formalising programming logic. At the beginning, I
was new to dependent types and proof assistants; therefore, we
had to choose carefully what theorems to formalise. On one hand, the theorems
should be non-trivial enough such that a substantial amount of work is required
to be done. On the other hand, the theorems should not be too
difficult because I am only a beginner in this area. Finally, we
decided to go with the Automata Theory as its basic concepts were
explained in the course \textit{Model of Computation}. 


\subsection{Outline}
\par Section 2 will be a brief introduction on Agda and
dependent types. We will describe how Agda can be used as a proof
assistant by giving examples of formalised proofs. Experienced Agda
users can skip this section and start from section 3 directly. In
section 3, we will describe several researches
that are also related to the formalisation of Automata
Theory. Following the background, section 4 will be a detail description of our
work. We will walk through the two components of our Agda
formalisation. Note that the definitions,
theorems and proofs written in this section are extracted from our
Agda code. They may be different from
their usual mathematical forms in order to adapt to the environment of
Type Theory. In section 5, we will discuss two possible extensions
to our project: 1) Myhill-Nerode Theorem and 2) the Pumping
Lemma. After that, in section 6, we will evaluate the project as a
whole. Finally, the conclusions will be drawn. 