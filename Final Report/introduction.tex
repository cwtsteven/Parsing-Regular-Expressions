\chapter{Introduction}
\par This project aims to study the feasibility of formalising
Automata Theory \cite{aho1972} in Type Theory
\cite{martin1984} with the aid of a dependently-typed functional
programming language, Agda. Automata Theory is an extensive work;
therefore, it will be unrealistic to
include all the materials under the time constraints. Accordingly,
this project will only focus on the theorems and
proofs that are related to the translation of regular expressions
to finite automata. In addition, this project also serves as an
example of how complex and non-trivial proofs are formalised. 

\par Our Agda formalisation consists of two components: 1) the
translation of regular expressions to minimal DFA and 2)
the correctness proofs of the translation. At this stage, we are only
interested in the correctness of the translation but not the
efficiency of the algorithms. 


\section{Background}
\par Type theory was introduced by Per Martin-L{\"o}f in
1971 to provide an alternative foundation of mathematics based on the
principles of mathematical constructivism where logic can be
implemented within the theory. From another perspective, Type Theory is also a
dependently-typed functional programming language. In order to bridge the gap between
the theoretical representation of Type Theory and the requirements on
a practical programming language, Norell \cite{norell2007} rewrote a
dependently-typed language, Agda, based on Type Theory. Agda allows us
to formalise mathematics and programming logic in Type Theory and to 
check the formalisation automatically by its type checker and termination
checker. We will discuss the use of Agda in Chapter 2 in detail. 

\par Automata Theory is the study of abstract machines and
automata. Automata are abstract models of machines that perform
computations on an input by moving between its states. There
are several major families of automaton but we will
only focus on finite state automata. 


\section{Related Work}
\paragraph{Regular Expressions in Agda} Agular and Mannaa published a
similar work \cite{agular2009} in 2009. They constructed a decider for
regular expressions which can determine whether
a given string is accepted by a given regular expression. The decider
was implemented by converting a regular expression into a
partial automaton. Here is the type signature of the decider: 
\begin{lstlisting}[mathescape=true,xleftmargin=.05\textwidth]
accept : (re : RegExp) $\to$ (as : List carrier) $\to$ Maybe (as $\in\smile [\![ re ]\!]$)
\end{lstlisting}
\par \(accept\) takes a regular expression \((re)\) and a list of
alphabets \((as)\) as the arguments. If the string \((as)\) is accepted
by the regular expression \((re)\), i.e. \mb{as
  \in L(re)}, the decider will return its proof. However it fails to generate a
proof for the opposite case, i.e. \mb{as \notin L(re)}. As they
explained in the paper, it is impossible without converting the regular expression into
the entire automaton. Therefore, in our approach, we will translate a regular
expression to a DFA in order to construct a full decider. 


\paragraph{Certified Parsing of Regular Languages in Agda} While in 2013, Firsov and Uustalu published another related
research \cite{firsov2013}. They translated regular expressions
into NFA and proved that they are equivalent. Unlike Agular and Mannaa's decider, Firsov and Uustalu's
algorithm can generate proofs for both cases. In their definition of NFA, the set of states
(\mb{Q}) and its subsets are represented as vectors; and the transition function
(\mb{\delta}) takes an alphabet as the argument and returns a matrix
representation of the transition table. 
\begin{lstlisting}[mathescape=true,xleftmargin=.3\textwidth]
record NFA : Set where
  field
    |Q| : $\mathbb N$
    $\delta$    : $\Sigma$ $\to$ |Q| $\ast$ |Q|
    I   : 1 $\ast$ |Q|
    F   : |Q| $\ast$ 1
\end{lstlisting}

\par This representation of sets allows us to
iterate the sets easily but it looks unnatural compare to the actual 
mathematical definition of NFA. Therefore, we intend
to avoid the vector representation in our approach. The comparison between these two
approaches will be discussed in Chapter 6 in detail. 


\section{Motivation}
\par My motivation on this project is to learn and apply
dependent types in formalising mathematics and programming logic. At the beginning, I
was new to dependent types and proof assistants; therefore, we
had to choose carefully what theorems to formalise. On one hand, the theorems
should be non-trivial enough such that a substantial amount of work is required
to be done. On the other hand, the theorems should not be too
difficult because I am only a beginner in this area. Finally, we
decided to go with the Automata Theory as its basic concepts were
explained in the course \textit{Model of Computation}. 


\section{Outline}
\par Chapter 2 will be a brief introduction on Agda and
dependent types. We will show how Agda can be used as a proof
assistant by formalising logic and small programs. Experienced Agda
users can skip this chapter. Following the background, Chapter 3 will be a detail description of our
work. We will walk through the two components of our Agda
formalisation. Note that the definitions,
theorems and proofs written in this chapter are extracted from our
Agda code. They may be different from
their usual mathematical forms in order to adapt to the environment of
Type Theory. In Chapter 4, we will describe the structure of our Agda
files. Then, in Chapter 5, we will discuss two possible extensions
to our project: 1) Myhill-Nerode Theorem and 2) the Pumping
Lemma. After that, in Chapter 6, we will evaluate the project as a
whole. Finally, the conclusions will be drawn. 