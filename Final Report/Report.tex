\documentclass[twoside,openright,final]{bhamthesis}
\usepackage[utf8]{inputenc}
\usepackage{amsmath}
\usepackage{listings}
\usepackage{geometry}
\usepackage{indentfirst}
\usepackage{enumerate,letltxmacro}
\LetLtxMacro\itemold\item
\renewcommand{\item}{\itemindent0.5cm\itemold}

\pagestyle{plain}

\setlength{\oddsidemargin}{1cm} % 2cm margin on the left for odd pages
\setlength{\evensidemargin}{0cm} % 2cm margin on the right for even pages

\title{Certified Parsing of Regular Expressions in Agda}
\department{Computer Science}
\degree{BSc. Computer Science}
\author{Wai Tak, Cheung}
\studentid{1465388}
\supervisor{Dr. Martín Escardó}

\begin{document}
\maketitle

\abstract
\par Blah blah blah. Blah blah blah. Blah blah blah. Blah blah blah. Blah
blah blah. Blah blah blah. Blah blah blah. Blah blah blah.
Blah blah blah. Blah blah blah. Blah blah blah. Blah blah blah. Blah
blah blah.
Blah blah blah. Blah blah blah. Blah blah blah. Blah blah blah.
Blah blah blah. Blah blah blah. Blah blah blah. Blah blah blah. \\ \\
Keywords: language, regular expression, finite automata, agda,
thompson's construction, powerset construction, proofs

\acknowledgments
\par Blah blah blah. Blah blah blah. Blah blah blah. Blah blah blah. Blah
blah blah. Blah blah blah. Blah blah blah.
Blah blah blah. Blah blah blah. Blah blah blah.
Blah blah blah. Blah blah blah. Blah blah blah. Blah blah blah. Blah
blah blah. Blah blah blah. Blah blah blah.
Blah blah blah. Blah blah blah. Blah blah blah. Blah blah blah. Blah
blah blah. Blah blah blah.
Blah blah blah. Blah blah blah. Blah blah blah. Blah blah blah. Blah
blah blah.

\repository
\vspace{7cm}
\begin{center}
  All software for this project can be found at \\
  https://codex.cs.bham.ac.uk/svn/projects/2015/wtc488/
\end{center}

\newpage
\tableofcontents
\newpage

\section{List of Abbreviations}
\begin{tabular}{ll}
  \textbf{\(\epsilon\)-NFA} & Non-deterministic finite automata with
                              \(\epsilon\)-transition \\
  \textbf{NFA} & Non-deterministic finite automata \\
  \textbf{DFA} & Deterministic finite automata 
\end{tabular}

\section{Introduction}
This work aims at implementing the parts in Formal Language Theory and
Automata Theory that are related to regular languages and finite automata
using Agda. The project is separated into three parts: 1) translating
regular expressions to NFA and DFA, 2) proving the correctness of
the translation and 3) formalising the Myhill-Nerode Theorem.

\par Here will be a paragraph emphasising the importance of
parsing/automata. 

\par The next section will be an introduction on proof assistance and
a review on related reseach topic. After that, the thrid section will
be a detail description of our approach on the project followed by the
evaulation. Finally, we will draw the conclusions. 


\section{Literature Review}
\subsection{Curry-Howard Isomorphism}
Relationship between programs and proofs.
\subsection{Agda}
Brief introduction on how agda works as a proof assistant, and how to
write proofs in agda.
\subsection{Related Work}
The matrix representation.


\section{Agda Formalisation}
Let us recall the three objectives of the project: 1) translating
regular expressions to NFA and DFA, 2) proving the correctness of the translation 
and 3) formalising the Myhill-Nerode Theorem. In part 1), we followed Thompson's
construction algorithm to build an \(\epsilon\)-NFA from a
regular expression. Then we removed all its \(\epsilon\)-transitions by computing the
\(\epsilon\)-closure for each state. Finally, we used powerset
construction to determinise the automata. (Minimize?) 

\par In this section, we will walk through the agda formalisation of
each of these steps together with their correctness proofs. However,
before we can go into these steps, we will have to define a
representation of subset it plays a major role in the Formal Language theory. 

\subsection{Subsets and Decidable Subsets}
\paragraph{Agda} Please refers to Subset.agda and Subset/DecidableSubset.agda

\paragraph{Definition 1.1} Suppose \(A\) is a set, in Agda, we represents its subset as a unary function on
\(A\), i.e. \(Subset\ A = A \to Set\). \\

\par When declaring a subset, we can write \(SubA =
\lambda\ a \to\ ?\) in Agda, the \(?\) here corresponds to the
condition for an element of \(A\) to be included in \(SubA\). This construction is
very similar to set comprehension. For example, the subset 
\(\{a\ | \ a \in A,\ P(a)\}\) corresponds to \(\lambda\ a \to P\ a\)
in Agda. As we mentioned before, a function in the form of \(A \to
Set\) is also a predicate of \(A\). Therefore, Subset is also a unary
predicate of \(A\). Thus, the decidibilty of Subset will remain
unknown until it is proved. 

\paragraph{Definition 1.2} The other representation of subset is \(DecSubset\ A = A \to
Bool\). Unlike \(Subset\), its decidability is ensured by its
definition. \\

\par The two definitions has different purposes. \(Subset\) was used to represent \(Language\) as not every
language is decidable. For other parts in the project 
such as the subset of states of an automata, \(DecSubset\) was used
as the decidability is assumed in the definition. The two definition
are defined in Subset.agda and Subset/DecidableSubset.agda
respectively as stated before, operations such as membership (\(\in\)), subset
(\(\subseteq\)), superset (\(\supseteq\)) and eqaulity (\(=\)) can
also be found in the two files. 

\par Now, by using the subset representation, we can define languages, regular expressions and finite
autotmata. Their formalisation in this project followed tightly to the
definition from Aho, A. and Ullman, J. (1972). 

\subsection{Languages}
\paragraph{Agda} Please refers to Language.agda \\

\par Suppose we have a set of alphabet \(\Sigma\). In Agda, it will be a data type,
i.e. \(\Sigma : Set\). 

\paragraph{Definition 2.1} We first define \(\Sigma^*\) as the set of all
strings over \(\Sigma\). In our approach, it was defined as a list of
\(\Sigma\), i.e. \(\Sigma^* = List\ \Sigma\). For example, (\(A :: g ::
d :: a :: []\)) represents the string 'Agda' and an empty string will
be represented as the empty list \([]\). In this way, when running the
automata, we can pattern match on the input string. 

\paragraph{Definition 2.2} A language is a subset of
\(\Sigma^*\); in Agda, \(Language = Subset\ \Sigma^*\). 
\(Subset\) instead of \(DecSubset\) was used because not every language is decidable. 

\subsubsection{Operations on Languages}

\paragraph{Definition 2.3} If \(L_1\) and \(L_2\) are languages, then
the union of the two languages \(L_1\cup L_2\) is defined as \(\{w\  |\  w \in L_1\ or\ w \in
L_2\}\). In Agda, we defined it as \(L_1 \cup L_2 = \lambda\ w \to w \in L_1\ \uplus\ w
\in L_2\) where \(\uplus\) is the Sum type representing the
proposition \(OR\). 

\paragraph{Definition 2.4} If \(L_1\) and \(L_2\) are languages, then
the concatenation of the two languages 
\(L_1\bullet L_2\) is defined
as \(\{w\  |\  \exists u\in L_1.\ \exists v\in L_2.\ w = uv\}\). In
Agda, we defined it as \(L_1\bullet L_2 = \lambda\ w \to \exists[
u \in \Sigma^* ]\ \exists[ v \in \Sigma^* ] ( u \in L_1 \times v \in v \in
L_2 \times w \equiv u\ ++\ v ) \) where \(\times\) is the Product type
representing the proposition \(AND\) and \(\equiv\) represents the
equivalency of two data. 

\paragraph{Definition 1.5} If \(L\) is a language, then the closure of
L \(L\ast\) is defined as \( \bigcup_{n \in N} L^n \) where
\( L^n = L\bullet P^{n - 1} \) and \(L^0 = \{\epsilon\}\). Agda formalisation?

\subsection{Regular Languages and Regular Expressions}
\paragraph{Agda} Please refers to RegularExpression.agda

\paragraph{Definition 3.1} We define regular languages over
\(\Sigma\) inductively as follows:
\begin{enumerate}
  \item \(\O\) is a regular language
  \item \(\{\epsilon\}\) is a regular language
  \item \(\forall a\in\Sigma.\ \{a\}\) is a regular language
  \item if \(L_1\) and \(L_2\) are regular languages, then
    \begin{enumerate}
      \item \(L_1\cup L_2\) is a regular language
      \item \(L_1\bullet L_2\) is a regular language
      \item \(L_1\ast\) is a regular language
    \end{enumerate}
  \item nothing else is a regular language
\end{enumerate}
The equivalent Agda definition: 
\begin{lstlisting}[mathescape=true,aboveskip=0pt,belowskip=0pt]
data Regular : Language $\to$ Set$_1$ where
  nullL : Regular $\O$
  empty : Regular $[\![ \epsilon ]\!]$
  singleton : (a : $\Sigma$) $\to$ Regular $[\![ a ]\!]$
  union : $\forall$ $L_1$ $L_2$ $\to$ Regular $L_1$ $\to$ Regular $L_2$ $\to$ Regular ($L_1\ \cup\ L_2$)
  conca : $\forall$ $L_1$ $L_2$ $\to$ Regular $L_1$ $\to$ Regular $L_2$ $\to$ Regular ($L_1\ \bullet\ L_2$) 
  kleen : $\forall$ $L$   $\to$ Regular $L$   $\to$ Regular ($L\ \star$)
\end{lstlisting}

\paragraph{Definition 3.2} Here we define regular expressions over
\(\Sigma\) as follows: 
\begin{enumerate}
  \item \(\O\) is a regular expression denoting the regular language \(\O\)
  \item \(\epsilon\) is a regular expression denoting the regular language \(\{\epsilon\}\)
  \item \(\forall a\in\Sigma.\ a\) is a regular expression denoting the regular language \(\{a\}\)
  \item if \(e_{1}\) and \(e_{2}\) are regular expression denoting the regular
    languages \(L_1\) and \(L_2\) respectively, then
    \begin{enumerate}
      \item \(e_{1}\ |\ e_{2}\) is a regular expressions denoting the
        regular languag \(L_1 \cup L_2\)
      \item \(e_{1}\cdot e_{2}\) is a regular expression denoting the
        regular language \(L_1\bullet L_2\)
      \item \(e_{1}\ ^{*}\) is a regular expression denoting the regular
        language \(L_1\ast\) 
     \end{enumerate}
  \item nothing else is a regular expression 
\end{enumerate}

\subsection{Non-deterministic Finite Automata with
  \(\epsilon\)-transitions}
\paragraph{Agda} Please refers to module \textbf{\(\epsilon\)-NFA} in
Automata.agda \\

\par By now, every string we have considered are in the form of
\(List\ \Sigma^*\). However, this definition gives us no way to pattern match
an \(\epsilon\)-step in the automata. Therefore, we need to introduce another
set of alphabet that includes \(\epsilon\). (For Definition 4.1 and
4.2, please refers to Language.agda)

\paragraph{Definition 4.1} We define \(\Sigma^e\) as the union of
\(\Sigma\) and \(\{\epsilon\}\), i.e. \(\Sigma^e = \Sigma \cup
\{\epsilon\}\). In Agda, this is simply a datatype definition:
\begin{lstlisting}[mathescape=true,xleftmargin=.4\textwidth,aboveskip=0pt,belowskip=0pt]
data $\Sigma^e$ : Set where
  $\alpha$ : $\Sigma \to \Sigma^e$
  E : $\Sigma^e$
\end{lstlisting}

\paragraph{Definition 4.2} Now we define \(\Sigma^{e*}\), the set of all strings over
\(\Sigma^e\) in a way similar to \(\Sigma^*\), i.e. \(\Sigma^{e*} =
List\ \Sigma^e\). For example, the string 'Agda' can be
represented as (\(\alpha\ A :: \alpha\ g :: E :: \alpha\ d :: E :: \alpha\
a :: []\)). 

\paragraph{Definition 4.3} A \(\epsilon\)-NFA is a 5-tuple \(M = (Q ,
\Sigma, \delta, q_0, F)\), where

\subsection{Thompson's Construction}

\paragraph{Theorem 1.1} The language accepted by a regular expression
is equal to the language accepted by the translated \(\epsilon\)-NFA. 

\subsection{Non-deterministic Finite Automata without \(\epsilon\)-transitions}

\subsection{Removing \(\epsilon\)-transitions}

\subsection{Deterministic Finite Automata}

\subsection{Powerset Construction}

\subsection{Myhill-Nerode Theorem}

\section{Evaluation}
(?)

\section{Conclusion}
(?)

\section{References}
Aho, A. and Ullman, J. (1972). The Theory of Parsing, Translation and Compiling. Volume I: Parsing. United States of America: Prentice-Hall, Inc.

\section{Appendix}
Agda Code?

\end{document}