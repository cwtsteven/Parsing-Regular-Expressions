\section{Deterministic Finite Automata}
\par The type of DFA is defined in \textbf{DFA.agda} along with its
operations and properties. 

\begin{defn}
\noindent A DFA is a 5-tuple \(M = (Q
,\ \Sigma,\ \delta,\ q_0,\ F)\), where
\begin{enumerate}[nolistsep]
  \item \(Q\) is a finite set of states;
  \item \(\Sigma\) is the set of alphabets;
  \item \(\delta\) is a mapping from \(Q \times \Sigma\) to \(Q\) that defines the behaviour of the automata;
  \item \(q_0\) in \(Q\) is the initial state;
  \item \(F \subseteq Q\) is the set of accepting states. 
\end{enumerate}
\end{defn}

\par It is formalised as a record in Agda as shown below: 

\begin{lstlisting}[mathescape=true,xleftmargin=.15\textwidth]
record DFA : Set$_1$ where
  field
    Q      : Set
    $\delta$       : Q $\to$ $\Sigma$ $\to$ DecSubset Q
    q$_0$      : Q
    F      : DecSubset Q
    _$\approx$_     : Q $\to$ Q $\to$ Set
    $\approx\hyphen$isEquiv : IsEquivalence _$\approx$_
    $\delta\hyphen$lem    : $\forall$ {q} {p} a $\to$ q $\approx$ p $\to$ $\delta$ q a $\approx$ $\delta$ p a
    F$\hyphen$lem   : $\forall$ {q} {p}   $\to$ q $\approx$ p $\to$ q $\in^d$ F $\to$ p $\in^d$ F
\end{lstlisting}

\par The set of alphabets \(\Sigma\) is passed into the module as a
parameter. Together with \(Q\), \(\delta\),
\(q_0\) and \(F\), these five fields correspond to the 5-tuple
DFA. The other extra fields are used in proving its decidability. They
are \(\_\approx\_\) -- an equivalence relation on states;
\(\approx\hyphen isEquiv\) -- a proof that the relation \(\approx\) is
an equivalence relation; \(\delta\hyphen lem\) -- a proof that for any
alphabet \(a\) and any two states \(q\) and \(p\), if \(q \approx p\)
then \(\delta (q,a) \approx \delta (p,a)\); and \(F\hyphen lem\) -- a
proof that for any two states \(q\) and \(p\), if \(q \approx p\) and
\(q\) is an accepting state, then \(p\) is also an accepting state. 

\par Now, before we can define the accepting language of a given
DFA, we need to define several operations of DFA. 

\begin{defn}
\noindent A configuration is composed of a state and an alphabet from
\(\Sigma\), i.e. \(C = Q \times \Sigma\). 
\end{defn}

\begin{defn}
\noindent A move in an DFA is
represented by a binary function (\(\vdash\)) on two configurations. We say
that for all \(w \in \Sigma^*\) and \(a \in \Sigma\), \((q, aw)
\vdash (q' , w)\) if and only if \(q' = \delta (q , a)\). 
\end{defn}

\par The binary function is defined in Agda as follow: 
\begin{lstlisting}[mathescape=true,xleftmargin=.1\textwidth]
  _$\vdash$_ : (Q $\times$ $\Sigma$ $\times$ $\Sigma^*$) $\to$ (Q $\times$ $\Sigma^*$) $\to$ Set
  (q , a , w) $\vdash$ (q' , w') = w $\equiv$ w' $\times$ q' $\approx$ $\delta$ q a
\end{lstlisting}

\begin{defn}
\noindent Suppose \(C\) and \(C'\) are configurations. We say that \(C \vdash^0 C'\) if and only
if \(C = C'\); and \(C_0 \vdash^k C_k\) for any \(k \geq 1\) if and only if there exists a chain of
configurations \(C_1, C_2, ..., C_{k-1}\) such that \(C_i \vdash C_{i+1}\) for all \(0 \leq i < k\). 
\end{defn}

\par It is defined as a recursive function in Agda as follow: 
\begin{lstlisting}[mathescape=true,xleftmargin=.1\textwidth]
  _$\vdash^k$_-_ : (Q $\times$ $\Sigma^*$) $\to$ $\mathbb{N}$ $\to$ (Q $\times$ $\Sigma^*$) $\to$ Set
  (q , w) $\vdash^k$ zero  - (q' , w')
    = q $\equiv$ q' $\times$ w $\equiv$ w'
  (q , w) $\vdash^k$ suc n - (q' , w') 
    = $\exists$[ p $\in$ Q ] $\exists$[ a $\in$ $\Sigma$ ] $\exists$[ u $\in$ $\Sigma^*$ ]
      (w $\equiv$ a :: u $\times$ (q , a , u) $\vdash$ (p , u) 
       $\times$ (p , u) $\vdash^k$ n - (q' , w'))
\end{lstlisting}

\begin{defn}
\noindent We say that \(C \vdash^* C'\) if and only
if there exists a number of chains \(n\) such that \(C \vdash^n C'\). 
\end{defn}

\par Its corresponding type is defined as follow: 
\begin{lstlisting}[mathescape=true,xleftmargin=.1\textwidth]
  _$\vdash^*$_ : (Q $\times$ $\Sigma^*$) $\to$ (Q $\times$ $\Sigma^*$) $\to$ Set
  (q , w) $\vdash^*$ (q' , w') = $\exists$[ n $\in$ $\mathbb{N}$ ] (q,w) $\vdash^k$ n - (q',w')
\end{lstlisting}

\begin{defn}
\noindent For any string \(w\), it is accepted by an DFA
if and only \(w\) can take \(q_0\) to an accepting state \(q\). Therefore, the
accepting language of an DFA is given by the set \(\{w\ |\ \exists q\in F.\ (q_0,w) \vdash^* (q,\epsilon)\}\). 
\end{defn}

\par The corresponding formalisation in Agda is as follow: 
\begin{lstlisting}[mathescape=true,xleftmargin=.1\textwidth]
  L$^D$ : DFA $\to$ Language
  L$^D$ dfa = $\lambda$ w $\to$ 
            $\exists$[ q $\in$ Q ] (q $\in^d$ F $\times$ (q$_0$ , w) $\vdash^*$ (q , [])))
\end{lstlisting} 


\section{Powerset Construction}
\par The translation of NFA to DFA is defined as the function
\textbf{powerset-construction} in the file \textbf{Translation/NFA-DFA.agda}. 

\begin{defn}
\label{defn:powerset}
\noindent For any given NFA, \((Q,\ \delta,\ q_0,\ F)\), its
translated DFA will be \((\mathcal P \left({Q}\right),\ \delta',\ \{q_0\},\ F')\) where
\begin{itemize}[nolistsep]
  \item \(\delta'(qs,a) = \{q'\ |\ \exists q\in qs.\ q' \in \delta (q,a)\}\);
  \item \(F' = \{qs\ |\ \exists q\in F.\ q \in qs\}\). 
\end{itemize}
\end{defn}

\par In Agda, the set \(\mathcal P \left({Q}\right)\) is defined as
the set of decidable subsets, i.e. \(Q' = DecSubset Q\). Therefore,
the decidability of the set \(\delta'(qs,a)\) and \(F'\) must be
proved using the vector representation of \(Q\). The corresponding
proofs are defined in the module \textbf{Powerset-Construction} in the
same file. 

\par Now, before proving that their accepting languages are equal, we 
need to prove the following lemmas which can be found in
\textbf{Correctness/NFA-DFA.agda}. 

\begin{lem}
\label{lem:nfa<dfa}
\noindent Let a NFA be \(N = (Q,\ \delta,\ q_0,\ F)\) and its
translated DFA be \(D = (\mathcal P \left({Q}\right),\ \delta',\ {q_0},
F')\) according to \autoref{defn:powerset}. For any string \(w\), if \(w\) can take \(q_0\) to a state
\(q\) with \(n\) transitions in \(N\), then there must exist a subset \(qs\) such that \(q
\in qs\) and \(w\) can take \(\{q_0\}\) to \(qs\) in \(D\),
i.e. \(\forall w.\exists q.\exists n.\ (q_0,w) \vdash^n (q,w) \Rightarrow
\exists qs.\ q \in qs \wedge ({q_0},w) \vdash^* (qs,\epsilon)\). 
\end{lem}

\par The proof is defined as the function \mmb{lem_1} under the module
\mmb{L^N\!\subseteq L^D}. 

\begin{proof}
\noindent We can prove the lemma by induction on \(n\).
\par \noindent \textbf{Base case.}\quad If \(n = 0\), then \(q_0 = q\)
and \(w = \epsilon\). It is obvious that the statement holds.

\par \noindent \textbf{Induction hypothesis.}\quad For any string \(w\), if \(w\) can take \(q_0\) to a state
\(q\) with \(k\) transitions in \(N\), then there exists a subset \(qs\) such that \(q
\in qs\) and \(w\) can take \(\{q_0\}\) to \(qs\) in \(D\). 

\par \noindent \textbf{Induction step.}\quad If \(n = k + 1\), then
\(w\) can take \(q_0\) to a state \(q\) by \(k + 1\) transitions. Let
\(w = w'a\) where \(a\) is an alphabet, then \(w'\) can take \(q_0\)
to a state \(p\) by \(k\) transitions and \(q \in \delta(p,a)\). By induction hypothesis, there must exist a subset \(ps\) such that \(p
\in ps\) and \(w'\) can take \(\{q_0\}\) to \(ps\) in \(D\). Furthermore, since \(q \in \delta(p,a)\), then \(a\) must be
able to take \(ps\) to a subset \(qs\) where \(q \in qs\). Therefore,
there exists a subset \(qs\) such that \(q \in qs\) and \(w\) can take
\(\{q_0\}\) to \(qs\); and thus the statement is true. 
\end{proof}

\begin{lem}
\label{lem:nfa>dfa}
\noindent Let a NFA be \(N = (Q,\ \delta,\ q_0,\ F)\) and its
translated DFA be \(D = (\mathcal P \left({Q}\right),\ \delta',\ {q_0},
F')\) according to \autoref{defn:powerset}. For any string \(w\), any
number \(n\) and any two states \(qs\) and \(ps\) in \(\mathcal P \left({Q}\right)\), if
\((qs,w) \vdash^n (ps,\epsilon)\) then \(ps =
\{p\ |\ \exists q\in qs.\ (q,w) \vdash^n (p,\epsilon)\}\). 
\end{lem}

\par The proof is defined as the function \mmb{lem_2} under the module
\mmb{L^N\!\supseteq L^D}. 

\begin{proof}
\noindent We can prove the lemma by induction on \(n\). 

\par \noindent \textbf{Base case.}\quad If \(n = 0\), then \(qs = ps\)
and \(w = \epsilon\). Then for any state \(p\) in \(Q\), 
\par \noindent  1) if \(p \in ps\), then \(p\) is also in \(qs\). It is obvious that
\((p,\epsilon) \vdash^0 (p, \epsilon)\) is true; therefore, \(p \in
\{p\ |\ \exists q\in qs.\ (q,w) \vdash^0 (p,\epsilon)\}\); 
\par \noindent  2) if \(p \in \{p\ |\ \exists q\in qs.\ (q,w) \vdash^0
(p,\epsilon)\}\), then \(p\) must be in \(qs\) and thus \(p \in ps\). 

\par \noindent \textbf{Induction hypothesis.}\quad For any string \(w\) and any
two states \(qs\) and \(ps\) in \(\mathcal P \left({Q}\right)\), if
\((qs,w) \vdash^k (ps,\epsilon)\) then \(ps =
\{p\ |\ \exists q\in qs.\ (q,w) \vdash^k (p,\epsilon)\}\). 

\par \noindent \textbf{Induction step.}\quad If \(n = k + 1\), then
\(w\) must be able to take \(qs\) to \(ps\) with \(k + 1\)
transitions in \(D\). Therefore there must exist an alphabet \(a\)
that can take \(qs\) to a subset \(rs\),
i.e. \(rs = \delta'(qs,a)\); and a string \(u\) that can take \(rs\) to
\(ps\) with \(k\) transitions. By induction hypothesis, we have \(ps =
\{p\ |\ \exists r\in rs.\ (r,u) \vdash^k (p,\epsilon)\}\). Then for any state \(p\) in \(Q\), 

\par \noindent  1) if \(p \in ps\), then there must exist a state \(r \in rs\)
such that \((r,u) \vdash^k (p,\epsilon)\). Since \(rs =
\delta'(qs,a)\); therefore, \(r \in \delta'(qs,a)\) and thus there
exists a state \(q \in qs\) such that \(r \in \delta (q,a)\). Therefore,
\((q,w) \vdash^{k+1} (p,\epsilon)\) is true and thus \(p \in \{p\ |\ \exists q\in qs.\ (q,w) \vdash^{k+1}
(p,\epsilon)\}\). 

\par \noindent  2) if \(p \in \{p\ |\ \exists q\in qs.\ (q,w) \vdash^{k+1}
(p,\epsilon)\}\), then there exists a state \(q \in qs\) such that \((q,w) \vdash^{k+1}
(p,\epsilon)\). Also, there must exist a state \(r \in Q\) such that
\(r \in \delta (q,a)\) and the string \(u\) can take \(r\) to \(p\) in
\(N\). Since, \(q
\in qs\) and \(r \in \delta (q,a)\); therefore, \(r \in \delta'(qs,a)
= rs\) and thus \(p \in \{p\ |\ \exists r\in rs.\ (r,u) \vdash^k
(p,\epsilon)\} = ps\). 
\end{proof}

\par Now, by using the above lemmas, we can prove the correctness of
the translation by proving that their accepting languages are
equal. The correctness proof is defined as the function \mmb{L^N
  \!\approx\! L^D} in \textbf{Correctness.agda} while the detail
proofs are defined in \textbf{Correctness/NFA-DFA.agda}. 

\begin{thm}
\noindent For any NFA, its accepting language is equal to
the language accepted by its translated DFA, i.e. \(L(\)NFA\()
= L(\)translated DFA\()\). 
\end{thm}

\begin{proof}
\noindent For a given NFA, \(N = (Q,\ \delta,\ q_0,\ F)\), let its
translated DFA be \(D = (\mathcal P \left({Q}\right),\ \delta',\
{q_0},\ F')\). To
prove the theorem, we have to prove that \(L(N) \subseteq L(D)\) and
\(L(N) \supseteq L(D)\). For any string \(w\), 

\par \noindent 1) if \(N\) accepts \(w\), then \(w\) can take \(q_0\) to an
accepting state \(q\) with \(n\) transitions in \(N\). By
\autoref{lem:nfa<dfa}, there must exist a subset \(qs\) such that
\(q \in qs\) and \(w\)
can take \(\{q_0\}\) to \(qs\) in \(D\). Since \(q\) is
an accepting state; therefore, \(qs\) is also an accepting state in
\(D\) and thus \(D\) accepts \(w\). 

\par \noindent 2) if \(D\) accepts \(w\), then \(w\) can take
\(\{q_0\}\) to an accepting state \(qs\) in \(D\) with \(n\)
transitions. Since \(qs\) is an accepting state, therefore, there must
exist a state \(q\) in \(Q\) such that \(q \in qs\) and \(q\) is also
an accepting state in \(N\). Assuming \(w\) cannot take \(q_0\) to
\(q\) in \(N\) for any number of transitions, then by
\autoref{lem:nfa>dfa}, \(qs = \emptyset\) and thus \(q \notin qs\) which contradicts the assumption that \(q \in
qs\). Therefore, \(w\) must be able to take \(q_0\) to \(q\) in \(N\) and thus \(N\) accepts \(w\). 
\end{proof}


\section{Decidability of DFA and Regular Expressions}
\par Recall that the accepting language of a DFA is given by the set
\(\{w\ |\ \exists q\in F.\ (q_0,w) \vdash^* (q,\epsilon)\}\), which is
equivalent to the set \(\{w\ |\ \exists q\in F.\ \exists n.\ (q_0,w) \vdash^n (q,\epsilon)\}\). 
Its decidability cannot be proved directly because the set of natural
numbers is infinite. Therefore, we have to
introduce another representation and prove
that they are equivalent. The representation and the related lemmas are
defined under the module \textbf{DFA-Operations} in \textbf{DFA.agda}. 

\begin{defn}
\noindent We define a function \(\delta^*(q,w)\) that takes a state \(q\) and
a string \(w\) as the arguments and returns a state \(p\) after
running the DFA. It is defined recursively as follow: 
\begin{lstlisting}[mathescape=true,xleftmargin=.3\textwidth]
$\delta^*$ : Q $\to$ $\Sigma^*$ $\to$ Q
$\delta^*$ q [] = q
$\delta^*$ (a :: w) = $\delta^*$ ($\delta$ q a) w
\end{lstlisting}
\end{defn}

\begin{defn}
\noindent We define \(\delta_0(w)\) as the function that runs the DFA
from \(q_0\) with a string \(w\). 
\begin{lstlisting}[mathescape=true,xleftmargin=.3\textwidth]
$\delta_0$ : $\Sigma^*$ $\to$ Q
$\delta_0$ w = $\delta^*$ $q_0$ w
\end{lstlisting}
\end{defn}

\par Now, before proving that the two representations are equivalent, we have to prove the following lemmas. 

\begin{lem}
\label{lem:dec_iff}
\noindent For any state \(q\) and any string \(w\), \(\delta^*(q,w) \in F\) if and only
if \(\exists q'\in F.\ \exists n.\ (q,w) \vdash^n (q',\epsilon)\).
\end{lem}

\begin{proof}
\noindent We have to prove for both directions. 

\par \noindent 1) If \(\delta^*(q,w) \in F\), we do induction on \(w\).
\par \textbf{Base case.}\quad If \(w = [\ ]\), then \(\delta^*(q,[\ ]) =
q \in F\). Therefore, the statement holds.
\par \textbf{Induction hypothesis.}\quad For any \(q\) and \(w'\), if
\(\delta^*(q,w') \in F\), then \(\exists q'\in F.\ \exists n.\ (q,w') \vdash^n (q',\epsilon)\).
\par \textbf{Induction step.}\quad If \(w = aw'\), then
\(\delta^*(q,aw') = \delta^*(\delta(q,a),w')\). By induction
hypothesis, there exists a state \(q' \in F\) and a number \(n\) such
that \((\delta(q,a),w') \vdash^n (q',\epsilon)\). It is equivalent to
\((q,aw') \vdash^{n+1} (q',\epsilon)\) and thus, the statement holds.

\par \noindent 2) If \(\exists q'\in F.\ \exists n.\ (q,w') \vdash^n
(q',\epsilon)\), then we do induction on \(n\).
\par \textbf{Base case.}\quad If \(n = 0\), then \(q' = q\) and \(w =
[\ ]\). Therefore, \(\delta^*(q,[\ ]) = q = q' \in F\). 
\par \textbf{Induction hypothesis.}\quad For any state \(q\) and
string \(w\), if
there exists another state \(q'\) and a number \(k\) such that \(q' \in F \wedge (q,w) \vdash^k
(q',\epsilon)\), then \(\delta^*(q,w) \in F\). 
\par \textbf{Induction step.}\quad If \(n = k + 1\), then there
exists a state \(q' \in F\) and a number \(k\) such that \((q,aw) \vdash^{k+1}
(q',\epsilon)\). It is equivalent to \((\delta(q,a),w) \vdash^k
(q',\epsilon)\). By induction hypothesis, we have
\(\delta^*(\delta(q,a),w) \in F\) and thus \(\delta^*(q,aw) \in F\). 
\end{proof} 

\par Now, we can prove that the two representations are equivalent. 

\begin{lem}
\label{lem:dec_iff2}
\noindent For any string
\(w\), \(\delta_0 (w) \in F\) if and only if \(\exists q\in F.\ (q_0,w) \vdash^* (q,\epsilon)\).
\end{lem}

\par The proof is defined as the function \mb{\delta_0\!-lem_1}. 

\begin{proof}
\noindent Since \(\delta_0(w) = \delta^*(q_0,w)\); therefore, by
\autoref{lem:dec_iff}, \(\delta_0(w) \in F\) if and only if \(\exists
q\in F.\ (q_0,w) \vdash^* (q,\epsilon)\). 
\end{proof}

\par Now, we can prove that the accepting language of a given DFA is
decidable by using the above theorems. The proof is
defined as the function \textbf{Dec-L\(^D\)} in \textbf{DFA.agda}. 

\begin{thm}
\noindent For any DFA, its accepting language is decidable,
i.e. \(\forall w.\ w \in L(\)DFA\()\) is decidable. 
\end{thm}

\begin{proof}
\noindent Since the language of DFA is given by the set \(\{w\ |\
\exists q\in F.\ (q_0,w) \vdash^* (q,\epsilon)\}\), which, by
\autoref{lem:dec_iff2}, is equal to the set \(\{w\ |\ \delta_0(w) \in
F\}\). Since \(F\) is a decidable subset; therefore, the set is also
decidable. 
\end{proof}

\par Since we have also proved that the accepting language of regular
expressions and DFA are equal; therefore, the accepting language of
regular expression must also be decidable. The proof is defined as the
function \textbf{Dec-L\(^R\)} in \textbf{RegExp-Decidability.agda}. 

\begin{thm}
\noindent For any given regular expression, \(e\), its accepting language is
decidable, i.e. \(\forall w.\ w \in L(\)e\()\) is decidable. 
\end{thm}

\begin{proof}
\noindent Since \(L(e) = L(translated\) DFA\()\) and the language
accepted by a DFA is decidable; therefore, \(L(e)\) is also decidable. 
\end{proof}