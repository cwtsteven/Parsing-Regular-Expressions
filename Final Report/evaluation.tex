\section{Evaluation}

\subsection{Correctness and Readability}
\paragraph{} According to Geuvers \cite{geuvers2009}, a proof has two major roles: 1)
to convince the readers that a statement is correct and 2) to expain
why a statement is correct. A proof is a verification of a statement
which is consist of smaller reasoning steps. These steps will eventually
contributes to the correctness of the proof. We can say that the a
proof is correct if and only if every reasoning step within the proof
is correct. The second part requires the proof to be able to give an
intuition of why the statement is correct. In the paragraphs below, we
will discuss these two criteria. 

\paragraph{Correctness} Traditionally, when a mathematician submits
the proof of his/her concepts, a group of mathematicians will
evaluate and check its correctness. Alternatively, if we
formalise the proof in a proof assistant, the proof will be checked
automatically by the compiler. The only difference is that we are now
relying on the compiler and the machine that runs the compile rather
than the group of mathematicians. Therefore, if the compiler and the
machine works properly, then any formalised proof that
can be compiled without errors are said to be correct. In our case, we
have the type checker and the termination checker in Agda to serve the
purpose. Furthermore, the correctness of a proof should be
obtained by verifiying the individual smaller reasoning steps within the
proof. When writing proofs in paper, we usually omit the proofs of
some obvious lemmas. However, in Agda, we have provide explicit proofs
to every lemma we have used. Therefore, the correctness of an Agda proof will always depend on the
correctness of the smaller proofs that it contains. 

\paragraph{Readability} The other purpose of a proof is to explain why
a certain statement is correct. However, in general, a computer proof
is incompetent on this purpose. 


\subsection{Different choices of representation}
\paragraph{} In this project, we have made several decisions over the
representation of mathematical objects. When writing proofs in papers,
we are not usually required to provide
concreate representations for abstract mathematical objects such as
subsets. However, when writing proofs in Type Theory, we usually have
to do that. The consequence is that different forms of represenation
will lead to different formalisations and thus contributes to the
easiness/difficulty in writing the proofs. In the following part, we
will discuss different representations we have choosen and their
effect. 

\paragraph{} In our approach,
we used \(Q : Set\) to represent the set of states while in
\cite{firsov2013}, they used vector as the representation. The 


\subsection{Problem arises}
\paragraph{} reachable states problem 
